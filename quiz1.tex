\documentclass[fleqn]{article}
\oddsidemargin 0.0in
\textwidth 6.0in
\thispagestyle{empty}
\usepackage{import}
\usepackage{amsmath}
\usepackage{graphicx}
\usepackage{flexisym}
\usepackage{amssymb}
\usepackage{bigints} 
\usepackage[english]{babel}
\usepackage[utf8x]{inputenc}
\usepackage{float}
\usepackage[colorinlistoftodos]{todonotes}

\definecolor{hwColor}{HTML}{AD53BA}

\begin{document}

  \begin{titlepage}

    \newcommand{\HRule}{\rule{\linewidth}{0.5mm}}

    \center


    \textsc{\LARGE Arizona State University}\\[1.5cm]

    \textsc{\LARGE Advanced Calculus I }\\[1.5cm]


    \begin{figure}
      \includegraphics[width=\linewidth]{asu.png}
    \end{figure}


    \HRule \\[0.4cm]
    { \huge \bfseries Quiz One }\\[0.4cm] 
    \HRule \\[1.5cm]

    \textbf{Behnam Amiri}

    \bigbreak

    \textbf{Prof: Sergei Suslov}

    \bigbreak


    \textbf{{\large \today}\\[2cm]}

    \vfill

  \end{titlepage}

  \begin{enumerate}
    \item (6 points)  If $x, y$ and $b$ are greater then zero and $\dfrac{x}{y}<\dfrac{a}{b}$
    , prove that $\dfrac{x}{y} < \dfrac{x+a}{y+b} < \dfrac{a}{b}$.

      \textcolor{hwColor}{
        On top of the given assumption, we also assume that $a$ is positive.
        \\
        \\
        $
          \dfrac{x}{y} < \dfrac{a}{b}
          \\
          \\
          \dfrac{x}{y} \times b < \dfrac{a}{b} \times b
          \\
          \\
          \dfrac{xb}{y} < a
          \\
          \\
          \dfrac{xb}{y} \times y < a \times y
          \\
          \\
          xb < ay 
          \\
          \\
          xb+ab < ay+ab
          \\
          \\
          b(x+a) < a(y+b)
          \\
          \\
          \dfrac{b(x+a)}{(y+b)} < \dfrac{a(y+b)}{(y+b)}
          \\
          \\
          \\
          \therefore ~~~~ \dfrac{(x+a)}{(y+b)} < \dfrac{a}{b} ~~~~ \checkmark
        $
        \\
        \\
        We just showed that $\dfrac{(x+a)}{(y+b)} < \dfrac{a}{b}$ and because we assumed $\dfrac{x}{y}<\dfrac{a}{b}$ then 
        $\dfrac{x}{y} < \dfrac{x+a}{y+b} < \dfrac{a}{b}$ holds.
        \\
        \\
      }

    \item (6 points) Prove that for all $n \in \mathcal{N}$,
    $$
      \sum\limits_{k=1}^{n} (2k-1)=1+3+5+...+(2n-1)=n^2
    $$
    [Hint: Use the Principle of Mathematical Inductio]

      \textcolor{hwColor}{
        \\
        \textbf{Basis: $k=1$}
        \\
        \\
        $
          1=1 \Longrightarrow L.H.S=R.H.S ~~~~ \checkmark 
        $
        \\
        \\
        \textbf{Induction:}
        \\
        \\
        Assume $1+2+...+n=\dfrac{n(n+1)}{2}$ is true for some $n=w$. Therefore, $1+2+...+w=\dfrac{w(w+1)}{2}$.
        \\
        \\
        Now we need to show that the statement is also true for $n=w+1$.
        \\
        \\
        $
          1+2+...+w+(w+1)=\dfrac{(w+1)\left[(w+1)+1\right]}{2}
          \\
          \\
          \\
          \therefore ~~~~ \dfrac{w(w+1)}{2}+(w+1)=\dfrac{(w+1)(k+2)}{2}
          \\
          \\
          \\
          \therefore ~~~~ \dfrac{w(w+1)}{2}+\dfrac{2(w+1)}{2}=\dfrac{(w+1)(w+2)}{2}
          \\
          \\
          \\
          \therefore ~~~~ w(w+1)+2(w+1)=(w+1)(w+2)
          \\
          \\
          \\
          \therefore ~~~~ w^2+w+2w+2=w^2+2w+w+2 ~~~~ \checkmark
        $
        \\
        \\
        So we basically showed that this is true for $n=w+1$.
      }


    \item (5 points) Suppose that $x_n$ converges to $0$. Show that $\dfrac{1}{x_n}$ is not bounded. 
    [Hint: Use a proof by contradiction.]

      % \textcolor{hwColor}{

      % }


    \item (3 points each) Evaluate the following limits when they exists [Hint: You may use the
    properties of limits.]
    \begin{enumerate}
      \item $\lim\limits_{n\to\infty} \dfrac{n^3+12n^2-n+1}{2n^3-7n^2+2n-12}$

        % \textcolor{hwColor}{

        % }


      \item $\lim\limits_{n\to\infty} \dfrac{\dfrac{1}{n^77}-2n^7+1}{\dfrac{2}{n^77}+5n^7-1}$
    
        % \textcolor{hwColor}{

        % }

      \item $\lim\limits_{n\to\infty} \dfrac{\left[cos(n)\right]^2}{n^2+1}$

        % \textcolor{hwColor}{

        % }

    \end{enumerate}
    

    \item (6 points) Show that for any sets $A, B,$ and $C$ the following property holds: 
    $A \setminus B=A$ if and only if $A \cap B=\emptyset$. 
    [Hint: By the definition $A \setminus B=\{x: x \in A ~ \& ~ x \notin B\}$.]

      % \textcolor{hwColor}{

      % }


    \item (6 points) Suppose that a sequence $\{a_n\}_{n=1}^{\infty}$ converges to $a$, namely,
    $\lim\limits_{n\to\infty} a_n=a$, and that $a_n>0$ for every $n$. Show that $a\geq 0$. 
    [Hint: Suppose that $a<0$ and take $\epsilon=-\dfrac{a}{2}$. Show that $\{a_n\}_{n=1}^{\infty}$ cannot converge to $a$.]

      % \textcolor{hwColor}{

      % }


    \item (6 points) Suppose that $\{a_n\}_{n=1}^{\infty}$ and $\{b_n\}_{n=1}^{\infty}$ are sequences with $\lim\limits_{n\to\infty} a_n=L$.
    Show that if $\lim\limits_{n\to\infty} (a_n-b_n)=0$, then $\lim\limits_{n\to\infty} b_n=L$.

      % \textcolor{hwColor}{

      % }

    \item (6 points) Prove that the sequence $\{\dfrac{n+1}{n}\}_{n=1}^{\infty}$ is Cauchy.

      % \textcolor{hwColor}{

      % }


    \item (Extra credit, 5 points) Find the limit of the sequence $\{a_n\}_{n=1}^{\infty}$ with the general term given by 
    $a_n=\left(\sqrt{4-\dfrac{1}{n}}-2\right)n$.

      % \textcolor{hwColor}{

      % }

  \end{enumerate}

\end{document}
