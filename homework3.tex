\documentclass[fleqn]{article}
\oddsidemargin 0.0in
\textwidth 6.0in
\thispagestyle{empty}
\usepackage{import}
\usepackage{amsmath}
\usepackage{graphicx}
\usepackage{flexisym}
\usepackage{amssymb}
\usepackage{bigints} 
\usepackage[english]{babel}
\usepackage[utf8x]{inputenc}
\usepackage{float}
\usepackage[colorinlistoftodos]{todonotes}

\definecolor{hwColor}{HTML}{AD53BA}

\begin{document}

  \begin{titlepage}

    \newcommand{\HRule}{\rule{\linewidth}{0.5mm}}

    \center


    \textsc{\LARGE Arizona State University}\\[1.5cm]

    \textsc{\LARGE Advanced Calculus I }\\[1.5cm]


    \begin{figure}
      \includegraphics[width=\linewidth]{asu.png}
    \end{figure}


    \HRule \\[0.4cm]
    { \huge \bfseries Homework 3 }\\[0.4cm] 
    \HRule \\[1.5cm]

    \textbf{Behnam Amiri}

    \bigbreak

    \textbf{Prof: Sergei Suslov}

    \bigbreak


    \textbf{{\large \today}\\[2cm]}

    \vfill

  \end{titlepage}

  \textbf{1.4}
  \begin{enumerate}
    \item \textbf{(36)} Let $\{a_n\}_{n=1}^{\infty}$ be a bounded sequence of real numbers. Prove that
    $\{a_n\}_{n=1}^{\infty}$ has a convergent subsequence. (Hint: You may want to use the Bolzano-Weierstrass Theorem.)

    \textcolor{hwColor}{
       We are told that the sequence is bounded meaning, it is both bounded below and bounded above. $(a_n \leq M, ~~ a_n\geq N)$. 
       \\
       \\
       Bolzano's theorem states that if we have a continuous function $f: [a,b] \longrightarrow \mathbb{R}$ is continuous function
       such that $f(a)f(b)<0$, then there is $c \in (a,b)$ such that $f(c)=0$.  
       \\
       \\
       Since this sequence is boubded we can split its interval up into two halfs. Because $\{a_n\}_{n=1}^{\infty}$ has
       infinitely many terms, we know one of these intervals must contain infinitely many terms as well. From the one sub-interval
       we pick, choose a term. Now just repeat what we did. (splitting up into two intervals).
       We can keep making these nested intervals that are getting smaller and smaller but every single time 
       they contain infinitely many terms so as we keep picking, the numbers are getting infinitely close to one another
       which means they must approach something. Therefore, all bounded sequences have convergent subsequence.
    }

    \item \textbf{(40)} Show that the sequence defined by $a_1=6$ and $a_n=\sqrt{6+a_{n-1}}$ for $n > 1$ is convergent and find the limit.
  
      \textcolor{hwColor}{
        If any given sequence $\{a_n\}$ is bounded and monotonic, then the sequence is convergent.
        \\
        \\
        \textbf{Monotonic:}
        \\
        \\
        When for a sequence $a_n \geq a_{n+1}$ then we have a monotonic decreasing sequence.
        \\
        \\
        $
          a_{n+2} \geq a_{n+1} \Longrightarrow \sqrt{6+a_{n+1}} \geq \sqrt{6+a_n} ~~~~ \checkmark
          \\
          \\
          \\
          \lim\limits_{n \to \infty} a_n=L \Longrightarrow \lim\limits_{n \to \infty} a_{n+1}=\sqrt{6+L}
          \\
          \\
          \\
          \therefore ~~~~ L=\sqrt{6+L} 
          \\
          \\
          \\
          \therefore ~~~~ L=3 ~~~~ \checkmark
        $
      }
  \end{enumerate}


  \rule{15cm}{2pt}

  \textbf{2.1}
  \begin{enumerate}
    \item \textbf{(1)} Define $f: (-2, 0) \longrightarrow R$ by $f(x)=\dfrac{x^2-4}{x+2}$. Prove that $f$ has a limit 
    at $-2$, and find it.

      \textcolor{hwColor}{
        If for every number $\epsilon > 0$ there is a number $\delta$ such that if 
        $$
          0<|x-a|< \delta, ~ \textrm{then} ~ |f(x)-L| < \epsilon
        $$
        Let's choose $\epsilon > 0$ so we should have $0 < |x-(-2)| < \delta$. We can easily 
        show that $L=-4$. But we assume that's the case. Then we have:
        \\
        \\
        $
          |f(x)-L|=|\dfrac{x^2-4}{x+2}-(-4)|=|x-2|< \delta
        $
        \\
        \\
        Suppose $\delta = \epsilon$. Therefore, the limit of $f(x)$ is $-4$ at $-2$. $~~~~ \checkmark$
      }
    
    
    \item \textbf{(2)} Define $f: (-2, 0) \longrightarrow R$ by $f(x)=\dfrac{2x^2+3x-2}{x+2}$. Prove that $f$ has a limit 
    at $-2$, and find it.

      \textcolor{hwColor}{
        If for every number $\epsilon > 0$ there is a number $\delta$ such that if 
        $$
          0<|x-a|< \delta, ~ \textrm{then} ~ |f(x)-L| < \epsilon
        $$
        Let's choose $\epsilon > 0$ so we should have $0 < |x-(-2)| < \delta$. We can easily 
        show that $L=-5$. But we assume that's the case. Then we have:
        \\
        \\
        $
          |f(x)-L|=|\dfrac{2x^2+3x-2}{x+2}-(-5)|=|2x+4|=2|x+2|<2 \delta
          \\
          \\
          \delta=\dfrac{\epsilon}{2}
        $
        \\
        \\
        Therefore, the limit of $f(x)$ is $-5$ at $-2$. $~~~~ \checkmark$
      }


    \item \textbf{(5)} Suppose $f: D \longrightarrow R$ with $x_0$ an accumulation point of $D$. Assume $L_1$ and $L_2$ 
    are limits of $f$ at $x_0$. Prove $L_1=L_2$. (Use only the definition; in later theorems, this uniqueness is
    assumed.) 

      \textcolor{hwColor}{
        If for every number $\epsilon > 0$ there is a number $\delta$ such that if 
        $$
          0<|x-a|< \delta, ~ \textrm{then} ~ |f(x)-L| < \epsilon
        $$
        Let's choose $\epsilon > 0$. The goal is to show $|L_1-L_2|< \epsilon$. 
        \\
        \\
        \textbf{Suppose:}
        \\
        \\
        $
          \begin{cases}
            |f(x)-L_1| < \dfrac{\epsilon}{2}
            \\
            \\
            |f(x)-L_2| < \dfrac{\epsilon}{2}
          \end{cases}
        $
        \\
        \\
        Now we have:
        \\
        \\
        $
          |L_1-L_2|=|\left(L_1-f(x)\right)+\left(f(x)-L_2\right)| \leq |f(x)-L_1|+|f(x)-L_2| < \dfrac{\epsilon}{2}+\dfrac{\epsilon}{2}
        $
        \\
        \\
        $\epsilon$ is a small value, therefore $L_1=L_2 ~~~~ \checkmark$.
      }


    \item \textbf{(8)} Define $f: (0, 1) \longrightarrow R$ by $f(x)=\dfrac{x^3-x^2+x-1}{x-1}$. Prove that $f$ has a limit at $1$?

      \textcolor{hwColor}{
        Let the limit of the function be $2$ and let $\epsilon > 0$.
        \\
        \\
        $
          |f(x)-L|=|\dfrac{x^3-x^2+x-1}{x-1}-2|=|x^2-1|
        $
        \\
        \\
        \textbf{Upper-bound:}
        \\
        \\
        $
          \delta > 1, ~~ |x-1|<\delta<1 \Longrightarrow 1<x+1<3
          \\
          \\
          \\
          \therefore ~~~~ |x-1||x+1|<3 \delta=\epsilon
        $
        Hence, the limit of $f(x)$ is $2$ at $1$. $~~~~ \checkmark$
      }

  \end{enumerate}

  \rule{15cm}{2pt}

  \textbf{2.2}
  \begin{enumerate}
    \item \textbf{(10)} Consider $f: (0,2) \longrightarrow R$ defined by $f(x)=x^x$. Assume that $f$ has a limit at $0$ and find that limit.
    (Hint: Choose a sequence $\{x_n\}_{n=1}^{\infty}$ converging to $0$ such that the limit of the
    sequence $\{f(x_n)\}_{n=1}^{\infty}$ is easy to determine.)

      % \textcolor{hwColor}{
        
      % }


    \item \textbf{(12)} Suppose $f: D \longrightarrow R$ has a limit at $x_0$. Prove that $|f|: D \longrightarrow R$ has 
    a limit at $x_0$ and that $\lim\limits_{x \to x_0} |f(x)|=|\lim\limits_{x \to x_0} f(x)|$.

      % \textcolor{hwColor}{
        
      % }


    \item \textbf{(13)} Define $f: R \longrightarrow R$ by $f(x)=x-[x]$. (See Example 2.6 for the definition 
    of $[x]$.) Determine those points at which f has a limit, and justify your conclusions.

      % \textcolor{hwColor}{
        
      % }

  \end{enumerate}

  \rule{15cm}{2pt}

  \textbf{2.3}
  \begin{enumerate}
    \item \textbf{(16)} Define $f: (0, 1) \longrightarrow R$ by $f(x)=\dfrac{x^3+6x^2+x}{x^2-6x}$. Prove that $f$ 
    has a limit at $0$ and find that limit. 

      % \textcolor{hwColor}{
        
      % }


    \item \textbf{(18)} Define $f: (0, 1) \longrightarrow R$ by $g(x)=\dfrac{\sqrt{1+x}-1}{x}$. Prove that $g$ has a limit
    at $0$ and find it.

      % \textcolor{hwColor}{
        
      % }


    \item \textbf{(19)} Define $f: (0, 1) \longrightarrow R$ by $f(x)=\dfrac{\sqrt{9-x}-3}{x}$. Prove that $f$ has a limit 
    at $0$ and find it.

      % \textcolor{hwColor}{
        
      % }


    \item \textbf{(22)} Show by example that, even though $f$ and $g$ fail to have limits at $x_0$, it is possible for
    $f+g$ to have a limit at $x_0$. Give similar examples for $fg$ and $\dfrac{f}{g}$.

      % \textcolor{hwColor}{
        
      % }


  \end{enumerate}

  \rule{15cm}{2pt}

  \textbf{3.1}
  \begin{enumerate}
    \item \textbf{(1)} Define $R \longrightarrow R$ by $f(x)=3x^2-2x+1$. Show that $f$ is continuous at $2$.

      % \textcolor{hwColor}{
        
      % }


    \item \textbf{(2)} Define $f: [-4, 0] \longrightarrow R$ by $f(x)=\dfrac{2x^2-18}{x+3}$ for $x \neq -3$ and $f(-3)=-12$. Show 
    that $f$ is continuous at $-3$.

      % \textcolor{hwColor}{
        
      % }


    \item \textbf{(5)} Define $f: (0, 1) \longrightarrow R$ by $f(x)=\dfrac{1}{\sqrt{x}}-\sqrt{\dfrac{x+1}{x}}$. Can one define
    $f(0)$ to make $f$ continuous at $0$? Explain. 

      % \textcolor{hwColor}{
        
      % }


    \item \textbf{(6)} Prove that $f(x)=\sqrt{x}$ is continuous for all $x \geq 0$.

      % \textcolor{hwColor}{
        
      % }
    

    \item \textbf{(9)} Define $f: (0, 1) \longrightarrow R$ by $f(x)=x sin(\dfrac{1}{x})$. Can one define
    $f(0)$ to make $f$ continuous at $0$? Explain.

      % \textcolor{hwColor}{
        
      % }

  \end{enumerate}  

  \rule{15cm}{2pt}

  \textbf{3.2}
  \begin{enumerate}
    \item \textbf{(14)} If $f: D \longrightarrow R$ is continuous at $x_0 \in D$, prove that the function $|f|: D \longrightarrow R$
    such that $|f|(x)=|f(x)|$ is continuous at $x_0$.

      % \textcolor{hwColor}{
        
      % }


    \item \textbf{(16)} Assume the continuity of $f(x)=e^x$ and $g(x)=ln(x)$. Define $h(x)=x^x$ by $x^x=e^{x ln(x)}$. Show $h$ is continuous
    at $x_0$.

      % \textcolor{hwColor}{
        
      % }
    

    \item \textbf{(17)} Suppose $f: D \longrightarrow R$ with $f(x) \geq 0$ for all $x \in D$. Show that, if $f$ is continuous at $x_0$, then
    $\sqrt{f}$ is continuous at $x_0$.

      % \textcolor{hwColor}{
        
      % }

  \end{enumerate}

  \rule{15cm}{2pt}

  \textbf{3.4}
  \begin{enumerate}
    \item \textbf{(41)} Find an interval of length $1$ that contains a root of the equation $x e^x=1$.

      % \textcolor{hwColor}{
        
      % }


    \item \textbf{(43)} Suppose $f: [a, b] \longrightarrow R$ is continuous and $f(b) \leq y \leq f(a)$. Prove that there is
    $c \in [a, b]$ such that $f(c)=y$.

      % \textcolor{hwColor}{
        
      % }
    
  \end{enumerate}

\end{document}
