\documentclass[fleqn]{article}
\oddsidemargin 0.0in
\textwidth 6.0in
\thispagestyle{empty}
\usepackage{import}
\usepackage{amsmath}
\usepackage{graphicx}
\usepackage{flexisym}
\usepackage{amssymb}
\usepackage{bigints} 
\usepackage[english]{babel}
\usepackage[utf8x]{inputenc}
\usepackage{float}
\usepackage[colorinlistoftodos]{todonotes}

\definecolor{hwColor}{HTML}{AD53BA}

\begin{document}

  \begin{titlepage}

    \newcommand{\HRule}{\rule{\linewidth}{0.5mm}}

    \center


    \textsc{\LARGE Arizona State University}\\[1.5cm]

    \textsc{\LARGE Advanced Calculus I }\\[1.5cm]


    \begin{figure}
      \includegraphics[width=\linewidth]{asu.png}
    \end{figure}


    \HRule \\[0.4cm]
    { \huge \bfseries Homework One }\\[0.4cm] 
    \HRule \\[1.5cm]

    \textbf{Behnam Amiri}

    \bigbreak

    \textbf{Prof: Sergei Suslov}

    \bigbreak


    \textbf{{\large \today}\\[2cm]}

    \vfill

  \end{titlepage}

  \begin{enumerate}
    \item Simplify each expression, where possible, in the following list. Give the
    principal value of inverse trigonometric functions, where applicable. If no
    simplification is possible, then so state.

    \begin{itemize}
      \item $ln (x^3)+ln (x^4)$.

      \item $e^{\dfrac{ln (x)}{2}}$.

      \item $e^{ln 7-ln 8}$.

      \item $e^0$.

      \item $cos^{2}(x)-1$.

      \item $sin(-x)$.

      \item $tan(\dfrac{\pi}{4})$.

      \item $Arctan(1)$.

    \end{itemize}


    \item Solve for $x$ where possible
    \begin{itemize}
      \item $ln(2x+2)=3t$.

      \item $x^2+4x+3=0$.

      \item $x^3+2x^2+x=0$.

      \item $Arctan (x)=1$.

      \item $Arcsin (x)=\dfrac{\sqrt{2}}{2}$.

    \end{itemize}

    \item Find the indicated derivative of each of the following expression.
    \begin{itemize}
      \item $\dfrac{d}{dt} \left(e^{-3t}\right)$.

        \textcolor{hwColor}{
          \\
          $
            \dfrac{d}{dt} \left(e^{-3t}\right)=e^{-3t} \times ln(e) \times \dfrac{d}{dt} \left(-3t\right)
            \\
            \\
            \\
            \therefore ~~~~ \dfrac{d}{dt} \left(e^{-3t}\right)=-3e^{-3t} ~~~~ \checkmark
            \\
          $
        }

      \item $\dfrac{d}{dt} \left(\dfrac{3}{t}\right)$.

        \textcolor{hwColor}{
          \\
          $
            \dfrac{d}{dt} \left(\dfrac{3}{t}\right)=\dfrac{(0 \times t)-(1 \times 3)}{t^2}
            \\
            \\
            \\
            \therefore ~~~~ \dfrac{d}{dt} \left(\dfrac{3}{t}\right)=-\dfrac{3}{t^2} ~~~~ \checkmark
            \\
          $
        }

      \item $\dfrac{d}{dx} \left[sin(5x)\right]$.

        \textcolor{hwColor}{
          \\
          $
            \dfrac{d}{dx} \left[sin(5x)\right]=cos(5x) \times \dfrac{d}{dx} \left(5x\right)
            \\
            \\
            \\
            \therefore ~~~~ \dfrac{d}{dx} \left[sin(5x)\right]=5cos(5x) ~~~~ \checkmark
            \\
          $
        }

      \item $\dfrac{d}{dx} \left[tan(x)\right]$.

        \textcolor{hwColor}{
          \\
          $
            \dfrac{d}{dx} \left[tan(x)\right]=\dfrac{d}{dx} \left[\dfrac{sin(x)}{cos(x)}\right]
            =\dfrac{cos(x).cos(x)-\left[-sin(x)\right]sin(x)}{cos^2(x)}
            =\dfrac{cos^2(x)+sin^2(x)}{cos^2(x)}
            \\
            \\
            \\
            \therefore ~~~~ \dfrac{d}{dx} \left[tan(x)\right]=\dfrac{1}{cos^2(x)}=sec^2(x) ~~~~ \checkmark
          $
        }

      \item $\dfrac{d}{dt} \left[t cos(t)\right]$.

        \textcolor{hwColor}{
          \\
          $
            \dfrac{d}{dt} \left[t cos(t)\right]=cos(t)+\left[-sin(t)t\right]
            \\
            \\
            \\
            \therefore ~~~~ \dfrac{d}{dt} \left[t cos(t)\right]=cos(t)-tsin(t) ~~~~ \checkmark
            \\
          $
        }

      \item $\dfrac{d}{dt} \left[ln(2t)\right]$.

        \textcolor{hwColor}{
          \\
          $
            \dfrac{d}{dt} \left[ln(2t)\right]=\dfrac{1}{2t} \dfrac{d}{dt} (2t)
            \\
            \\
            \\
            \therefore ~~~~ \dfrac{d}{dt} \left[ln(2t)\right]=\dfrac{1}{t} ~~~~ \checkmark
            \\
          $
        }

      \item $\dfrac{d}{dx} \left[\left(2x+4\right)^{10}\right]$.

        \textcolor{hwColor}{
          \\
          $
            \dfrac{d}{dx} \left[\left(2x+4\right)^{10}\right]=10(2)(2x+4)^9
            \\
            \\
            \\
            \therefore ~~~~ \dfrac{d}{dx} \left[\left(2x+4\right)^{10}\right]=20\left(2x+4\right)^9 ~~~~ \checkmark
            \\
          $
        }

      \item $\dfrac{d}{dx} \left(9x^8+\dfrac{1}{x^2}\right)$.

        \textcolor{hwColor}{
          \\
          $
            \dfrac{d}{dx} \left(9x^8+\dfrac{1}{x^2}\right)=72x^7+\dfrac{(0 \times x^2)-(2x \times 1)}{x^4}
            \\
            \\
            \\
            \therefore ~~~~ \dfrac{d}{dx} \left(9x^8+\dfrac{1}{x^2}\right)=72x^7-\dfrac{2}{x^3} ~~~~ \checkmark
            \\
          $
        }

      \item $\dfrac{d}{dx} \left(\dfrac{1}{\sqrt{2x+1}}\right)$.

        \textcolor{hwColor}{
          \\
          $
            \dfrac{d}{dx} \left(\dfrac{1}{\sqrt{2x+1}}\right)=\dfrac{(0 \times \sqrt{2x+1})-\dfrac{d}{dx} \left[\sqrt{2x+1}\right]}{\left(\sqrt{2x+1}\right)^2}
            =\dfrac{-\dfrac{1}{\sqrt{2x+1}}}{\left(\sqrt{2x+1}\right)^2}
            \\
            \\
            \\
            \therefore ~~~~ \dfrac{d}{dx} \left(\dfrac{1}{\sqrt{2x+1}}\right)=-\dfrac{1}{\left(2x+1\right)^{3/2}} ~~~~ \checkmark
            \\
          $
        }

      \item $\dfrac{d}{dt} \left(3t^2+1\right)^{\dfrac{3}{2}}$.

        \textcolor{hwColor}{
          \\
          $
            \dfrac{d}{dt} \left(3t^2+1\right)^{\dfrac{3}{2}}=\dfrac{3}{2} \left(6t\right) \left(3t^2+1\right)^{\dfrac{1}{2}}
            \\
            \\
            \\
            \therefore ~~~~ \dfrac{d}{dt} \left(3t^2+1\right)^{\dfrac{3}{2}}=9t \left(3t^2+1\right)^{\dfrac{1}{2}} ~~~~ \checkmark
            \\
          $
        }

      \item $\dfrac{d}{dt} \left(\dfrac{1}{\sqrt[3]{t+1}}\right)$.

        \textcolor{hwColor}{
          \\
          $
            \dfrac{d}{dt} \left(\dfrac{1}{\sqrt[3]{t+1}}\right)
            =\dfrac{d}{dt} \left(\dfrac{1}{(t+1)^{1/3}}\right)
            =\dfrac{d}{dt} \left[\left(t+1\right)^{-1/3}\right]
            =-\dfrac{1}{3} (1) \left(t+1\right)^{-4/3}
            \\
            \\
            \\
            \therefore ~~~~ \dfrac{d}{dt} \left(\dfrac{1}{\sqrt[3]{t+1}}\right)=-\dfrac{1}{3\left(t+1\right)^{4/3}} ~~~~ \checkmark
            \\
          $
        }

      \item $\dfrac{d}{dt} \left(2t+1\right)^{\dfrac{1}{4}}$.

        \textcolor{hwColor}{
          \\
          $
            \dfrac{d}{dt} \left(2t+1\right)^{\dfrac{1}{4}}=\dfrac{1}{4} (2) \left(2t+1\right)^{-\dfrac{3}{4}}
            \\
            \\
            \\
            \therefore ~~~~ \dfrac{d}{dt} \left(2t+1\right)^{\dfrac{1}{4}}=\dfrac{1}{2 \left(2t+1\right)^{3/4}} ~~~~ \checkmark
            \\
          $
        }

    \end{itemize}

    \item Evaluate each of the following limits. If the limit does not exist, then so
    state. Note that some limits may be $+\infty$ or $-\infty$; an infinite limit is not the
    same as a limit that does not exist. Give the principal value of the limits of
    inverse trigonometric functions, where applicable.
    \begin{itemize}
      \item $\lim\limits_{t \to \infty} e^{-t}$.

        \textcolor{hwColor}{
          \\
          \textbf{Option 1:}
          \\
          $
            \lim\limits_{t \to \infty} e^{-t}=\lim\limits_{t \to \infty} \dfrac{1}{e^t}=\dfrac{\lim\limits_{t \to \infty} 1}{\lim\limits_{t \to \infty} e^t}=\dfrac{1}{\infty}
            \\
            \\
            \\
            \therefore ~~~~ \lim\limits_{t \to \infty} e^{-t}=0 ~~~~ \checkmark 
          $
          \\
          \\
          \rule{15cm}{1pt}
          \\
          \\
          \textbf{Option 2:}
          \\
          $
            \lim\limits_{t \to \infty} e^{-t}
            =e^{\lim\limits_{t \to \infty} \left(-t\right)}
            =e^{-\lim\limits_{t \to \infty} t}
            =e^{-\infty}
            \\
            \\
            \\
            \therefore ~~~~ \lim\limits_{t \to \infty} e^{-t}=0 ~~~~ \checkmark
            \\
          $
          \\
          \\
          \includegraphics[width=10cm, height=6cm]{Twelve.JPG}
          \\
          \\
        }
      
      \item $\lim\limits_{x \to \infty} e^{2-x}$.

        \textcolor{hwColor}{
          \\
          $
            \lim\limits_{x \to \infty} e^{2-x}
            =\lim\limits_{x \to \infty} \left[e^2 e^{-x}\right]
            =e^2 \lim\limits_{x \to \infty} e^{-x}
            =e^2 \times 0
            \\
            \\
            \\
            \therefore ~~~~ \lim\limits_{x \to \infty} e^{2-x}=0 ~~~~ \checkmark
            \\
          $
          \\
          \\
          \includegraphics[width=10cm, height=6cm]{Eleven.JPG}
          \\
          \\
        }

      \item $\lim\limits_{x \to \infty} e^{\dfrac{1}{x}}$.
      
        \textcolor{hwColor}{
          \\
          $
            \lim\limits_{x \to \infty} e^{\dfrac{1}{x}}
            =e^{\lim\limits_{x \to \infty} \dfrac{1}{x}}
            =e^{\dfrac{\lim\limits_{x \to \infty} 1}{\lim\limits_{x \to \infty} x}}
            =e^{\dfrac{1}{\infty}}
            =e^0
            \\
            \\
            \\
            \therefore ~~~~ \lim\limits_{x \to \infty} e^{\dfrac{1}{x}}=1 ~~~~ \checkmark
            \\
          $
          \\
          \\
          \includegraphics[width=10cm, height=6cm]{Ten.JPG}
          \\
          \\
        }

      \item $\lim\limits_{x \to 0} e^{-x}$. 

        \textcolor{hwColor}{
          \\
          $
            \lim\limits_{x \to 0} e^{-x}=e^{-0}
            \\
            \\
            \\
            \therefore ~~~~ \lim\limits_{x \to 0} e^{-x}=1 ~~~~ \checkmark
            \\
          $
          \\
          \\
          \includegraphics[width=10cm, height=6cm]{Nine.JPG}
          \\
          \\
        }

      \item $\lim\limits_{x \to 0} \left(x e^{-x}\right)$.

        \textcolor{hwColor}{
          \\
          $
            \lim\limits_{x \to 0} \left(x e^{-x}\right)=0 \times e^{-0}
            \\
            \\
            \\
            \therefore ~~~~ \lim\limits_{x \to 0} \left(x e^{-x}\right)=0  ~~~~ \checkmark
            \\
          $
          \\
          \\
          \includegraphics[width=10cm, height=6cm]{Eight.JPG}
          \\
          \\
        }
      
      \item $\lim\limits_{x \to 0} tan(x)$. 

        \textcolor{hwColor}{
          \\
          $
            \lim\limits_{x \to 0} tan(x)=\lim\limits_{x \to 0} \dfrac{sin(x)}{cos(x)}
            =\dfrac{\lim\limits_{x \to 0} sin(x)}{\lim\limits_{x \to 0} cos(x)}
            =\dfrac{sin(0)}{cos(0)}=\dfrac{0}{1}
            \\
            \\
            \\
            \therefore ~~~~ \lim\limits_{x \to 0} tan(x)=0 ~~~~ \checkmark
            \\
          $
          \\
          \\
          \includegraphics[width=10cm, height=6cm]{Seven.JPG}
          \\
          \\
        }

      \item $\lim\limits_{x \to 0} \dfrac{sin(x)}{x}$. 

        \textcolor{hwColor}{
          \\
          We can apply the l'Hopital's rule since $\dfrac{0}{0}$ is indeterminate. \\
          $
            \lim\limits_{x \to 0} \dfrac{sin(x)}{x}
            =\lim\limits_{x \to 0} \dfrac{\dfrac{d}{dx}\left[sin(x)\right]}{\dfrac{d}{dx}(x)}
            =\lim\limits_{x \to 0} \dfrac{cos(x)}{1}=\lim\limits_{x \to 0} cos(0)
            \\
            \\
            \\
            \therefore ~~~~ \lim\limits_{x \to 0} \dfrac{sin(x)}{x}=1 ~~~~ \checkmark
            \\
          $
          \\
          \\
          \includegraphics[width=10cm, height=6cm]{Six.JPG}
          \\
          \\
        }

      \item $\lim\limits_{x \to 0} \dfrac{cos(x)}{x}$. 

        \textcolor{hwColor}{
          \\
          We know that the Taylor series expansion of $cos(x)$ is 
          $
            \sum\limits_{n=0}^{\infty} (-1)^n \dfrac{x^{2n}}{(2n)!}
          $
          \\
          \\
          \textbf{Limit from right:}
          \\
          \\
          $
            \lim\limits_{x \to 0^+} \dfrac{cos(x)}{x}
            =\lim\limits_{x \to 0^+} \dfrac{1-\dfrac{x^2}{2!}+\dfrac{x^4}{4!}-\dfrac{x^6}{6!}+\dfrac{x^8}{8!}-...}{x}
            =\lim\limits_{x \to 0^+} \dfrac{1}{x}-\dfrac{x}{2}+\dfrac{x^3}{4!}-\dfrac{x^5}{6!}+...
            \\
            \\
          $
          When $x$ approaches to $0$, all the terms from the 2nd onwards become $0$. 
          Therefore, the only term left is the first term, which is $\lim\limits_{x \to 0^+} \dfrac{1}{x}$. This 
          leaves us with $+\infty$.
          \\
          \\
          $
            \therefore ~~~~ \lim\limits_{x \to 0^+} \dfrac{cos(x)}{x}=+\infty ~~~~ \checkmark
          $
          \\
          \\
          \rule{15cm}{2pt}
          \\
          \\
          \textbf{Limit from left:}
          \\
          \\
          $
            \lim\limits_{x \to 0^-} \dfrac{cos(x)}{x}
            =\lim\limits_{x \to 0^-} \dfrac{1-\dfrac{x^2}{2!}+\dfrac{x^4}{4!}-\dfrac{x^6}{6!}+\dfrac{x^8}{8!}-...}{x}
            =\lim\limits_{x \to 0^-} \dfrac{1}{x}-\dfrac{x}{2}+\dfrac{x^3}{4!}-\dfrac{x^5}{6!}+...
          $
          \\
          \\
          When $x$ approaches to $0$, all the terms from the 2nd onwards become $0$. 
          Therefore, the only term left is the first term, which is $\lim\limits_{x \to 0^+} \dfrac{1}{x}$. This 
          leaves us with $-\infty$.
          \\
          \\
          $
            \therefore ~~~~ \lim\limits_{x \to 0^-} \dfrac{cos(x)}{x}=-\infty ~~~~ \checkmark
          $
          \\
          \\
          \\
          Since $\lim\limits_{x \to 0^-} \dfrac{cos(x)}{x} \neq \lim\limits_{x \to 0^+} \dfrac{cos(x)}{x}$, 
          then $\lim\limits_{x \to 0} \dfrac{cos(x)}{x}$ \textbf{does not exist!}
          \\
          \\
          \includegraphics[width=10cm, height=6cm]{One.JPG}
          \\
          \\
        }

      \item $\lim\limits_{x \to 0} \dfrac{x}{3x^2+1}$.

        \textcolor{hwColor}{
          \\
          $
            \lim\limits_{x \to 0} \dfrac{x}{3x^2+1}=\dfrac{0}{3(0)^2+1}
            \\
            \\
            \\
            \therefore ~~~~ \lim\limits_{x \to 0} \dfrac{x}{3x^2+1}=0 ~~~~ \checkmark
            \\
          $
          \\
          \\
          \includegraphics[width=10cm, height=6cm]{Two.JPG}
          \\
          \\
        }
      
      \item $\lim\limits_{x \to \infty} \dfrac{x}{3x^2+1}$. 

        \textcolor{hwColor}{
          \\
          $
            \lim\limits_{x \to \infty} \dfrac{x}{3x^2+1}
            =\lim\limits_{x \to \infty} \dfrac{\dfrac{1}{x}}{3+\dfrac{1}{x^2}}
            =\dfrac{\lim\limits_{x \to \infty} \left(\dfrac{1}{x}\right)}{\lim\limits_{x \to \infty} \left(3+\dfrac{1}{x^2}\right)}
            =\dfrac{0}{3}
            \\
            \\
            \\
            \therefore ~~~~ \lim\limits_{x \to \infty} \dfrac{x}{3x^2+1}=0 ~~~~ \checkmark
            \\
          $
          \\
          \\
          \includegraphics[width=10cm, height=6cm]{Two.JPG}
          \\
          \\
        }

      \item $\lim\limits_{x \to \infty} \dfrac{x^2+1}{x}$. 

        \textcolor{hwColor}{
          \\
          $
            \lim\limits_{x \to \infty} \dfrac{x^2+1}{x}
            =\lim\limits_{x \to \infty} \left(x+\dfrac{1}{x}\right)
            =\lim\limits_{x \to \infty} x+\lim\limits_{x \to \infty} \dfrac{1}{x}
            =\infty + 0
            \\
            \\
            \\
            \therefore ~~~~ \lim\limits_{x \to \infty} \dfrac{x^2+1}{x}=+\infty ~~~~ \checkmark
            \\
          $
          \\
          \\
          \includegraphics[width=10cm, height=6cm]{Five.JPG}
          \\
          \\
        }

      \item $\lim\limits_{x \to 0} \dfrac{x+1}{x-1}$. 

        \textcolor{hwColor}{
          \\
          $
            \lim\limits_{x \to 0} \dfrac{x+1}{x-1}=\dfrac{x+0}{0-1}
            \\
            \\
            \\
            \therefore ~~~~ \lim\limits_{x \to 0} \dfrac{x+1}{x-1}=-1 ~~~~ \checkmark
            \\
          $
          \\
          \\
          \includegraphics[width=10cm, height=6cm]{Four.JPG}
          \\
          \\
        }

      \item $\lim\limits_{x \to 1} \dfrac{x+1}{x-1}$.

        \textcolor{hwColor}{
          \\
          \textbf{Limit from left:}
          $
            \lim\limits_{x \to 1^-} \dfrac{x+1}{x-1}=-\infty
            \\
            \\
          $
          \\
          \textbf{Limit from right:}
          $
            \lim\limits_{x \to 1^+} \dfrac{x+1}{x-1}=+\infty
            \\
            \\
          $
          \\
          \\
          Since $\lim\limits_{x \to 1^-} \dfrac{x+1}{x-1} \neq \lim\limits_{x \to 1^+} \dfrac{x+1}{x-1}$
          then $\lim\limits_{x \to 1} \dfrac{x+1}{x-1}$ \textbf{does not exist!}
          \\
          \\
          \includegraphics[width=10cm, height=6cm]{Four.JPG}
          \\
          \\
        }
      
      \item $\lim\limits_{x \to \infty} \dfrac{3x^2+1}{4x^3+x}$. 

        \textcolor{hwColor}{
          \\
          $
            \lim\limits_{x \to \infty} \dfrac{3x^2+1}{4x^3+x}
            =\lim\limits_{x \to \infty} \dfrac{\dfrac{3}{x}+\dfrac{1}{x^3}}{4+\dfrac{1}{x^2}}
            =\dfrac{\lim\limits_{x \to \infty} \left(\dfrac{3}{x}+\dfrac{1}{x^3}\right)}{\lim\limits_{x \to \infty} \left(4+\dfrac{1}{x^2}\right)}
            =\dfrac{\lim\limits_{x \to \infty} \dfrac{3}{x}+\lim\limits_{x \to \infty} \dfrac{1}{x^3}}{\lim\limits_{x \to \infty} 4+ \lim\limits_{x \to \infty}\dfrac{1}{x^2}}
            =\dfrac{0+0}{4+0}
            \\
            \\
            \\
            \therefore ~~~~ \lim\limits_{x \to \infty} \dfrac{3x^2+1}{4x^3+x}=0 ~~~~ \checkmark
            \\
          $
          \\
          \\
          \includegraphics[width=10cm, height=6cm]{Three.JPG}
          \\
          \\
        }

    \end{itemize}

    \item Evaluate each of the following integrals. 
    \begin{itemize}
      \item $\bigints \dfrac{1}{x^3} dx$.

        % \textcolor{hwColor}{
        %   \\
        %   $
        %     \therefore ~~~~  ~~~~ \checkmark
        %     \\
        %   $
        % }

      \item $\bigints \dfrac{1}{3t+2} dt$.

      \item $\bigints e^{-3t} dt$.

      \item $\bigints t e^{-t} dt$.

      \item $\bigints \dfrac{x}{x^2+1} dx$.

      \item $\bigints \dfrac{x^2+1}{x} dx$.

      \item $\bigints \dfrac{1}{x^2+1}$.

      \item $\bigints\limits_{0}^{\dfrac{\pi}{4}} cos(2t) sin(2t) dt$.

      \item $\bigints\limits_{0}^{\dfrac{\pi}{4}} e^t cos(t) dt$.

      \item $\bigints\limits_{0}^{\dfrac{\pi}{4}} tan(x) dx$.

      \item $\bigints \dfrac{x}{\sqrt{x^2+1}} dx$.
    \end{itemize}


  \end{enumerate}

\end{document}
