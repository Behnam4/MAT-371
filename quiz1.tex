\documentclass[fleqn]{article}
\oddsidemargin 0.0in
\textwidth 6.0in
\thispagestyle{empty}
\usepackage{import}
\usepackage{amsmath}
\usepackage{graphicx}
\usepackage{flexisym}
\usepackage{amssymb}
\usepackage{bigints} 
\usepackage[english]{babel}
\usepackage[utf8x]{inputenc}
\usepackage{float}
\usepackage[colorinlistoftodos]{todonotes}

\definecolor{hwColor}{HTML}{AD53BA}

\begin{document}

  \begin{titlepage}

    \newcommand{\HRule}{\rule{\linewidth}{0.5mm}}

    \center


    \textsc{\LARGE Arizona State University}\\[1.5cm]

    \textsc{\LARGE Advanced Calculus I }\\[1.5cm]


    \begin{figure}
      \includegraphics[width=\linewidth]{asu.png}
    \end{figure}


    \HRule \\[0.4cm]
    { \huge \bfseries Quiz One }\\[0.4cm] 
    \HRule \\[1.5cm]

    \textbf{Behnam Amiri}

    \bigbreak

    \textbf{Prof: Sergei Suslov}

    \bigbreak


    \textbf{{\large \today}\\[2cm]}

    \vfill

  \end{titlepage}

  \begin{enumerate}
    \item (6 points)  If $x, y$ and $b$ are greater then zero and $\dfrac{x}{y}<\dfrac{a}{b}$
    , prove that $\dfrac{x}{y} < \dfrac{x+a}{y+b} < \dfrac{a}{b}$.

      \textcolor{hwColor}{
        On top of the given assumption, we also \textbf{have to} assume that $a$ is positive otherwise $\dfrac{x}{y}<\dfrac{a}{b}$ would not hold.
        \\
        \\
        $
          \dfrac{x}{y} < \dfrac{a}{b}
          \\
          \\
          \dfrac{x}{y} \times b < \dfrac{a}{b} \times b
          \\
          \\
          \dfrac{xb}{y} < a
          \\
          \\
          \dfrac{xb}{y} \times y < a \times y
          \\
          \\
          xb < ay 
          \\
          \\
          xb+ab < ay+ab
          \\
          \\
          b(x+a) < a(y+b)
          \\
          \\
          \dfrac{b(x+a)}{(y+b)} < \dfrac{a(y+b)}{(y+b)}
          \\
          \\
          \\
          \therefore ~~~~ \dfrac{(x+a)}{(y+b)} < \dfrac{a}{b} ~~~~ \checkmark
        $
        \\
        \\
        We just showed that $\dfrac{(x+a)}{(y+b)} < \dfrac{a}{b}$ and because we assumed $\dfrac{x}{y}<\dfrac{a}{b}$ then 
        $\dfrac{x}{y} < \dfrac{x+a}{y+b} < \dfrac{a}{b}$ holds.
        \\
        \\
      }

    \item (6 points) Prove that for all $n \in \mathcal{N}$,
    $$
      \sum\limits_{k=1}^{n} (2k-1)=1+3+5+...+(2n-1)=n^2
    $$
    [Hint: Use the Principle of Mathematical Inductio]

      \textcolor{hwColor}{
        \textbf{Basis: $n=1$}
        \\
        \\
        $
          1=1 \Longrightarrow L.H.S=R.H.S ~~~~ \checkmark 
        $
        \\
        \\
        \textbf{Induction:}
        \\
        \\
        Assume $1+3+5+...+(2n-1)=n^2$ is true for some $n=k$. 
        \\
        \\
        Therefore, $1+3+5+...+(2k-1)=k^2 ~~~~~ (A)$
        \\
        \\
        Now we need to show that the statement is also true for $n=k+1$.
        \\
        \\
        $
          1+3+5+...+ \left[2(k+1)-1\right]=(k+1)^2
          \\
          \\
          \\
          \therefore ~~~~ 1+3+5+...+\left(2k-1\right)+\left(2k+1\right)=(k+1)^2
          \\
          \\
          \\
        $
        From $(A)$, we know $1+3+5+...+(2k-1)=k^2$, hence:
        \\
        \\
        $
          \therefore ~~~~ k^2+\left(2k+1\right)=(k+1)^2
          \\
          \\
          \\
          \therefore ~~~~ \left(k+1\right)^2=\left(k+1\right)^2 ~~~~ \checkmark
        $
        \\
        \\
        So by induction, we showed that $1+3+5+...+(2n-1)=n^2$ holds.
      }


    \item (5 points) Suppose that $x_n$ converges to $0$. Show that $\dfrac{1}{x_n}$ is not bounded. 
    [Hint: Use a proof by contradiction.]

      \textcolor{hwColor}{
        \\
        When it comes to determining whether or not a sequence is bounded, the first thing that we need to determine before 
        we can determine whether is bounded, is whether or not the sequence monotonic or more specifically whether or not the sequence
        over the time is always increasing or always decreasing. 
      }


    \item (3 points each) Evaluate the following limits when they exists [Hint: You may use the
    properties of limits.]
    \begin{enumerate}
      \item $\lim\limits_{n\to\infty} \dfrac{n^3+12n^2-n+1}{2n^3-7n^2+2n-12}$

        \textcolor{hwColor}{
          \\
          $
            \lim\limits_{n\to\infty} \dfrac{n^3+12n^2-n+1}{2n^3-7n^2+2n-12}
            =\lim\limits_{n\to\infty} \dfrac{n^3+12n^2-n+1}{2n^3-7n^2+2n-12} \times \dfrac{\dfrac{1}{n^3}}{\dfrac{1}{n^3}}
            \\
            \\
            \\
            =\lim\limits_{n\to\infty} \dfrac{1+\dfrac{12}{n}-\dfrac{1}{n^2}+\dfrac{1}{n^3}}{2-\dfrac{7}{n}+\dfrac{2}{n^2}-\dfrac{12}{n^3}}=\dfrac{1+0-0+0}{2-0+0-0}
            \\
            \\
            \\
            \therefore ~~~~ \lim\limits_{n\to\infty} \dfrac{n^3+12n^2-n+1}{2n^3-7n^2+2n-12}=\dfrac{1}{2} ~~~~ \checkmark
          $
          \\
          \\
          \rule{15cm}{2pt}
        }

      \item $\lim\limits_{n\to\infty} \dfrac{\dfrac{1}{n^{77}}-2n^7+1}{\dfrac{2}{n^{77}}+5n^7-1}$
    
        \textcolor{hwColor}{
          \\
          $
            \lim\limits_{n\to\infty} \dfrac{\dfrac{1}{n^{77}}-2n^7+1}{\dfrac{2}{n^{77}}+5n^7-1}
            =\lim\limits_{n\to\infty} \dfrac{\dfrac{1}{n^{77}}-2n^7+1}{\dfrac{2}{n^{77}}+5n^7-1} \times \dfrac{n^{77}}{n^{77}}
            \\
            \\
            \\
            =\lim\limits_{n\to\infty} \dfrac{1-2n^{84}+n^{77}}{2+5n^{84}-n^{77}}
            =\lim\limits_{n\to\infty} \dfrac{1-2n^{84}+n^{77}}{2+5n^{84}-n^{77}} \times \dfrac{\dfrac{1}{n^{84}}}{\dfrac{1}{n^{84}}}
            \\
            \\
            \\
            =\lim\limits_{n\to\infty} \dfrac{\dfrac{1}{n^{84}}-2+\dfrac{1}{n^7}}{\dfrac{2}{n^{84}}+5-\dfrac{1}{n^7}}=\dfrac{0-2+0}{0+5-0}
            \\
            \\
            \\
            \therefore ~~~~ \lim\limits_{n\to\infty} \dfrac{\dfrac{1}{n^{77}}-2n^7+1}{\dfrac{2}{n^{77}}+5n^7-1}
            =\lim\limits_{n\to\infty} \dfrac{\dfrac{1}{n^{84}}-2+\dfrac{1}{n^7}}{\dfrac{2}{n^{84}}+5-\dfrac{1}{n^7}}=-\dfrac{2}{5} ~~~~ \checkmark
          $
          \\
          \\
          \rule{15cm}{2pt}
        }

      \item $\lim\limits_{n\to\infty} \dfrac{\left[cos(n)\right]^2}{n^2+1}$

        \textcolor{hwColor}{
          \\
          $
            \lim\limits_{n\to\infty} \dfrac{\left[cos(n)\right]^2}{n^2+1}=\lim\limits_{n\to\infty} \dfrac{\left[cos(n)\right]^2}{n^2+1} \times \dfrac{\dfrac{1}{n^2}}{\dfrac{1}{n^2}}
            \\
            \\
            \\
            =\lim\limits_{n\to\infty} \dfrac{\dfrac{\left[cos(n)\right]^2}{n^2}}{1+\dfrac{1}{n^2}}
            =\dfrac{\lim\limits_{n\to\infty} \dfrac{\left[cos(n)\right]^2}{n^2}}{\lim\limits_{n\to\infty} 1+\dfrac{1}{n^2}}
          $
          \\
          \\
          \\
          \textbf{Numerator:}
          \\
          \\
          We can use the squeeze theorem for this limit. We know that $cos(x)$ is always between $-1$ and $+1$.
          \\
          \\
          $
            0 \leq \left[cos(x)\right]^2 \leq +1 
            \\
            \\
            0 \leq \dfrac{\left[cos(x)\right]^2}{n^2} \leq \dfrac{1}{n^2} 
            \Longrightarrow \lim\limits_{n\to\infty} 0 \leq \lim\limits_{n\to\infty} \dfrac{\left[cos(x)\right]^2}{n^2} \leq \lim\limits_{n\to\infty} \dfrac{1}{n^2}
            \\
            \\
            \Longrightarrow 0 \leq \lim\limits_{n\to\infty} \dfrac{\left[cos(x)\right]^2}{n^2} \leq 0 
            \\
            \\
            \\
            \therefore ~~~~ \lim\limits_{n\to\infty} \dfrac{\left[cos(x)\right]^2}{n^2}=0 ~~~~ \checkmark
            \\
          $
          \\
          \\
          \textbf{Denominator:}
          \\
          \\
          $
            \lim\limits_{n\to\infty} 1+\dfrac{1}{n^2}=1+0=1 ~~~~ \checkmark
            \\
            \\
            \rule{15cm}{2pt}
          $
          \\
          \\
          $
            \therefore ~~~~ \dfrac{\lim\limits_{n\to\infty} \dfrac{\left[cos(n)\right]^2}{n^2}}{\lim\limits_{n\to\infty} 1+\dfrac{1}{n^2}}=\dfrac{0}{1}
            \\
            \\
            \\
            \therefore ~~~~ \lim\limits_{n\to\infty} \dfrac{\left[cos(n)\right]^2}{n^2+1}=\dfrac{\lim\limits_{n\to\infty} \dfrac{\left[cos(n)\right]^2}{n^2}}{\lim\limits_{n\to\infty} 1+\dfrac{1}{n^2}}=0 ~~~~ \checkmark
          $
        }

    \end{enumerate}
    

    \item (6 points) Show that for any sets $A, B,$ and $C$ the following property holds: 
    $A \setminus B=A$ if and only if $A \cap B=\emptyset$. 
    [Hint: By the definition $A \setminus B=\{x: x \in A ~ \& ~ x \notin B\}$.]

      \textcolor{hwColor}{
        \\
        Intuitively $A \setminus B$ represents the part of $A$ that is not in $B$ and $A \cap B$ represents the part of $A$ that is in $B$.
        When we combine the part of $A$ that is in $B$ and the part of $A$ that is not in $B$, we should just get $A$.
        \\
        \\
        We have $A$ and as two arbitrary sets. Let's have an element of set $A$ like $k$. In other words $k \in A$, then 
        for $A \setminus B=A$ only holds if $k \notin B$. Hence, $A \cap B=\emptyset$ meaning the two have nothing in common.
        \\
        \\
        \\
        Getting back to $A \cap B=\emptyset$. Let's have two elements $p$ and $q$ where $p \in A$ and $q \in B$. Then we have the following:
        \\
        \\
        $
          \begin{cases}
            q \notin A \setminus B
            \\
            \\
            p \in A \setminus B
          \end{cases} ~~~~ \checkmark
        $
        \\
        \\
        \\
        $
          \left(A \setminus B\right) \cup \left(A \cap B\right)
          =\left(A \cap B^C\right) \cup \left(A \cap B\right)
          =A \cap \left(B^C \cup B\right)
          =A \cap U
          =A
          \\
          \\
          \\
          A \setminus B \cup \left(A \cap B\right)=\left(A \setminus B\right) \cup \emptyset=A ~~~~ \checkmark
        $
        \\
        \\
        Since $A \cap B=\emptyset$, they have nothing in common. Thus assign some member to $A$ 
        and some different member to $B$. subtracting $B$ from $A$ would only leave the members of $A$.
        \\
      }


    \item (6 points) Suppose that a sequence $\{a_n\}_{n=1}^{\infty}$ converges to $a$, namely,
    $\lim\limits_{n\to\infty} a_n=a$, and that $a_n>0$ for every $n$. Show that $a \geq 0$. 
    [Hint: Suppose that $a<0$ and take $\epsilon=-\dfrac{a}{2}$. Show that $\{a_n\}_{n=1}^{\infty}$ cannot converge to $a$.]

      \textcolor{hwColor}{
        \\
        Let's think that the truth or the validity of the proposition, by showing that assuming the proposition 
        to be false leads to a contradiction. In our case we have $a \leq 0$. We are told that 
        $\{a_n\}_{n=1}^{\infty}$ converges to $a$ therefore $|a_n-a| < |a|$. 
        \\
        \\
        Since we assume that $a \leq 0$, this means that the postulate contradicts that $a_n >0$, hence we have proved that
        $a \geq 0$ is true. $~~~~ \checkmark$
      }


    \item (6 points) Suppose that $\{a_n\}_{n=1}^{\infty}$ and $\{b_n\}_{n=1}^{\infty}$ are sequences with $\lim\limits_{n\to\infty} a_n=L$.
    Show that if $\lim\limits_{n\to\infty} (a_n-b_n)=0$, then $\lim\limits_{n\to\infty} b_n=L$.

      \textcolor{hwColor}{
        Let's have $\dfrac{\epsilon}{2}>0$.
        \\
        \\
        $
          \therefore ~~~~ \begin{cases}
            |a_n-L|< \dfrac{\epsilon}{2}
            \\
            \\
            |a_n-b_n-0|< \dfrac{\epsilon}{2}
          \end{cases}
          \\
          \\
          \\
          |b_n-L|=|b_n-a_n+a_n-L| \leq |a_n-L|+|a_n-b_n|< \dfrac{\epsilon}{2}+\dfrac{\epsilon}{2}
          \\
          \\
          \\
          \therefore ~~~~ |b_n-L|<\epsilon
          \\
          \\
          \\
          \therefore ~~~~ \lim\limits_{n\to\infty} b_n=L ~~~~ \checkmark
        $
      }

    \item (6 points) Prove that the sequence $\{\dfrac{n+1}{n}\}_{n=1}^{\infty}$ is Cauchy.

      \textcolor{hwColor}{
        Let $\epsilon>0$ and $N \in \mathbb{N}$ such that $N >\dfrac{1}{\epsilon}$. Also let $m>n\geq N$.
        \\
        \\
        $
          |-\dfrac{m}{m+1}+\dfrac{n}{n+1}|=|\dfrac{nm+n-nm+m}{(m+1)(n+1)}|
          =|\dfrac{n-m}{(m+1)(n+1)}|
          \\
          \\
          \\
        $ 
        Since $mn<(m+1)(n+1)$ then $\dfrac{1}{mn}>\dfrac{1}{(m+1)(n+1)}$
        \\
        \\
        \\
        $
          |\dfrac{n}{(n+1)}-\dfrac{m}{(m+1)}|<|\dfrac{m}{(m+1)(n+1)}|<|\dfrac{m}{mn}|=\dfrac{1}{n}<\epsilon
        $
        \\
        \\
        \\
        Therefore, $\{\dfrac{n+1}{n}\}_{n=1}^{\infty}$ is Cauchy. $~~~~ \checkmark$
      }


    \item (Extra credit, 5 points) Find the limit of the sequence $\{a_n\}_{n=1}^{\infty}$ with the general term given by 
    $a_n=\left(\sqrt{4-\dfrac{1}{n}}-2\right)n$.

      \textcolor{hwColor}{
        \\
        $
          \left(\sqrt{4-\dfrac{1}{n}}-2\right)n=\left(\sqrt{4-\dfrac{1}{n}}-2\right)n \times \dfrac{\left(\sqrt{4-\dfrac{1}{n}}+2\right)n}{\left(\sqrt{4-\dfrac{1}{n}}+2\right)n}
          \\
          \\
          \\
          \therefore ~~~~ \left(\sqrt{4-\dfrac{1}{n}}-2\right)n=-\dfrac{1}{\sqrt{4-\dfrac{1}{n}}+2}
        $
        Let's make some assumption in order to use some formulas from the textbook.
        \\
        \\
        $
          \begin{cases}
            a_n=-1 \Longrightarrow \{a_n\}_{n=1}^{\infty} ~~~ \textrm{converges to -1}
            \\
            \\
            k_n=2 \Longrightarrow \{k_n\}_{n=1}^{\infty} ~~~ \textrm{converges to 2}
            \\
            \\
            d_n=\{\dfrac{1}{n}\}_{n=1}^{\infty} ~~~ \textrm{converges to 2}
          \end{cases}
        $
        \\
        \\
        \\
        On page 45 od the textbook we have a theorem stating:
        \\
        \\
        "If $\{a_n\}_{n=1}^{\infty}$ converges to $A$ and $\{b_n\}_{n=1}^{\infty}$
        converges to $B$, then $\{a_n+b_n\}_{n=1}^{\infty}$ converges to $A+B$". 
        \\
        \\
        By using this theorem we have:
        \\
        \\
        $
          \{-\dfrac{1}{\sqrt{4-\dfrac{1}{n}}+2}\}_{n=1}^{\infty}
        $ converges to $-\dfrac{1}{4}. ~~~~ \checkmark$
      }

  \end{enumerate}

\end{document}
