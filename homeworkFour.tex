\documentclass[fleqn]{article}
\oddsidemargin 0.0in
\textwidth 6.0in
\thispagestyle{empty}
\usepackage{import}
\usepackage{amsmath}
\usepackage{graphicx}
\usepackage{flexisym}
\usepackage{amssymb}
\usepackage{bigints} 
\usepackage[english]{babel}
\usepackage[utf8x]{inputenc}
\usepackage{float}
\usepackage[colorinlistoftodos]{todonotes}

\definecolor{hwColor}{HTML}{042c4d}
\setlength\parindent{24pt}

\begin{document}

  \begin{titlepage}

    \newcommand{\HRule}{\rule{\linewidth}{0.5mm}}

    \center


    \textsc{\LARGE Arizona State University}\\[1.5cm]

    \textsc{\LARGE Advanced Calculus I }\\[1.5cm]


    \begin{figure}
      \includegraphics[width=\linewidth]{asu.png}
    \end{figure}


    \HRule \\[0.4cm]
    { \huge \bfseries Homework Four }\\[0.4cm] 
    \HRule \\[1.5cm]

    \textbf{Behnam Amiri}

    \bigbreak

    \textbf{Prof: Sergei Suslov}

    \bigbreak


    \textbf{{\large \today}\\[2cm]}

    \vfill

  \end{titlepage}

  \textbf{4.1 THE DERIVATIVE OF A FUNCTION }
  \begin{enumerate}
    \item (2) Prove that the definition of the derivative and the alternate definition of the derivative are
    equivalent.

        \textcolor{hwColor}{
          \hspace{15pt} Let's start off by remembering what the definition of the derivative and the alternate definition are.
          \\
          \\
          \textbf{The definition of the derivative:}
          \\
          \hspace{15pt} The function $f$ is said to be differentiable at $x_0$ iff $\dfrac{f(x)-f(x_0)}{x-x_0}$ has a limit at $x_0$
          and we write $$\lim\limits_{x \to x_0} \dfrac{f(x)-f(x_0)}{x-x_0}=f^'(x_0)$$
          \\
          \textbf{The alternate definition of the derivative:}
          \\
          \hspace{15pt} The function $f$ is said to be differentiable at $x_0$ iff $\dfrac{f(x_0+t)-f(x_0)}{t}$ has a limit at 
          zero. If this limit exists, it is called the derivative of $f$ at $x_0$.
          \\
          \\
          Even though the two definitions seem different but if we look carefully they are the same indeed.
          \\
          \\
          \emph{\textbf{Proof:}}
          \\
          \\
          Suppose $x=x_0+t$ then we can have the following:
          \\
          \\
          $
            \lim\limits_{t \to 0} \dfrac{f(x_0+t)-f(x_0)}{t}
            \\
            \\
            =\lim\limits_{x-x_0 \to 0} \dfrac{f(x)-f(x_0)}{x-x_0}
            \\
            \\
            =\lim\limits_{x \to x_0} \dfrac{f(x)-f(x_0)}{x-x_0}
            \\
            \\
            \\
            \therefore ~~~~ \lim\limits_{t \to 0} \dfrac{f(x_0+t)-f(x_0)}{t}=\lim\limits_{x \to x_0} \dfrac{f(x)-f(x_0)}{x-x_0} ~~~~~~~~~ \blacksquare
            \\
          $
        }


    \item (3) Use the definition to find the derivative of $f(x)=\sqrt{x}$, for $x > 0$. Is $f$ differentiable at zero? Explain.

    \item (4) Use the definition to find the derivative of $g(x)=x^2$.

    \item (6) Suppose $f: (a, b) \longrightarrow R$ is differentiable at $x \in (a, b)$. Prove that 
    $$
      \lim\limits_{h \to 0} \dfrac{f(x+h)-f(x-h)}{2h}
    $$
    exists and equals $f^'(x)$. Give an example of a function where this limit exists, but the
    function is not differentiable.


  \end{enumerate}

  \rule{15cm}{1pt}

  \textbf{4.2  THE ALGEBRA OF DERWATZVES}
  \begin{enumerate}
    \item (11) Prove $f: (0, 1) \longrightarrow R$ defined by $f(x)=\sqrt{2x^2-3x+6}$ is differentiable on $(0, 1)$ and 
    compute the derivative.

    \item (14) Suppose $f: R \longrightarrow R$ is differentiable and define $g(x)=x^2 f(x^3)$. Show that $g$ is
    differentiable and compute $g^'$.
    
    \item (15) Define $f(x)=\sqrt{x+\sqrt{x+\sqrt{x}}}$ for $x \geq 0$. Determine where $f$ is differentiable and compute
    the derivative. 
  \end{enumerate}

  \rule{15cm}{1pt}

  \textbf{4.3 ROLLE'S THEOREM AND THE MEAN-VALUE THEOREM}
  \begin{enumerate}
    \item (16) Define $f: [0, 2] \longrightarrow R$ by $f(x)=\sqrt{2x-x^2}$. Show that $f$ satisfies the conditions of Rolle's
    theorem and find $c$ such that $f^'(c)=0$.

    \item (17) Define $f: R \longrightarrow R$ by $f(x)=\dfrac{1}{1+x^2}$. Prove that $f$ has a maximum value and find the
    point at which that maximum occurs.  

    \item (28) Prove that the function $f(x)=2x^3+3x^2-36x+5$ is $1-1$ on the interval $[-1,1]$. Is $f$ increasing or decreasing?
  \end{enumerate}

  \rule{15cm}{1pt}

  \textbf{4.4 L'HOSPITAL'S RLILE AND THE INVERSE-FUNCTION THEOREM}
  \begin{enumerate}
    \item Assume the rules for differentiating the elementary functions, and use L'Hospital's Rule
    and find the following limits: 
    \begin{enumerate}
      \item $\lim\limits_{x \to 1} \dfrac{ln(x)}{x-1}$

      \item $\lim\limits_{x \to 0} \dfrac{x}{e^x-1}$

      \item $\lim\limits_{x \to 0} \dfrac{sin(x)}{x}$
    \end{enumerate}

    \item (33) Use L'Hospitd's Rule to find the limit:
    $$
      \lim\limits_{x \to 0} \dfrac{x^2 sin(x)}{sin(x)-x cos(x)}
    $$

    \item (39) Suppose $f: R \longrightarrow R$ is such that $f(x+y)=f(x)f(y), f$ is differentiable at zero, and $f$ is
    not identically zero. Prove that $f$ is differentiable everywhere and that $f^'(x)=f(x)f^'(0)$.
    Assuming the properties of the exponential function, prove that $f(x)=e^{cx}$ where $c=f^'(0)$.

  \end{enumerate}
\end{document}
