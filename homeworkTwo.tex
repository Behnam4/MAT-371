\documentclass[fleqn]{article}
\oddsidemargin 0.0in
\textwidth 6.0in
\thispagestyle{empty}
\usepackage{import}
\usepackage{amsmath}
\usepackage{graphicx}
\usepackage{flexisym}
\usepackage{amssymb}
\usepackage{bigints} 
\usepackage[english]{babel}
\usepackage[utf8x]{inputenc}
\usepackage{float}
\usepackage[colorinlistoftodos]{todonotes}

\definecolor{hwColor}{HTML}{AD53BA}

\begin{document}

  \begin{titlepage}

    \newcommand{\HRule}{\rule{\linewidth}{0.5mm}}

    \center


    \textsc{\LARGE Arizona State University}\\[1.5cm]

    \textsc{\LARGE Advanced Calculus I }\\[1.5cm]


    \begin{figure}
      \includegraphics[width=\linewidth]{asu.png}
    \end{figure}


    \HRule \\[0.4cm]
    { \huge \bfseries Homework Two }\\[0.4cm] 
    \HRule \\[1.5cm]

    \textbf{Behnam Amiri}

    \bigbreak

    \textbf{Prof: Sergei Suslov}

    \bigbreak


    \textbf{{\large \today}\\[2cm]}

    \vfill

  \end{titlepage}

  \textbf{0.1}
  \begin{enumerate}
    \item List the elements of each of the following sets:
    \begin{enumerate}
      \item $\mathcal{J} \cap [0,6)$

        \textcolor{hwColor}{
          We know that $\mathcal{J}$ denotes the set of positive integers. so we have:
          \\
          \\
          $
            \mathcal{J} \cap [0,6)=\{1, 2, 3, 4, 5\} ~~~~ \checkmark
          $
          \\
        }

      \item $\mathcal{Z} \cap (-6, 2]$

        \textcolor{hwColor}{
          We know that $\mathcal{Z}$ denotes the set of integers. so we have:
          \\
          \\
          $
            \mathcal{Z} \cap (-6, 2]=\{-5, -4, -3, -2, -1, 0, 1, 2\} ~~~~ \checkmark
          $
          \\
        }

      \item $\{1, 2, 3, 4 \}  \cup \{2, 3, 4, 5\}$

        \textcolor{hwColor}{
          \\
          $
            \{1, 2, 3, 4, 5\} ~~~~ \checkmark
          $
          \\
        }

      \item $\{1, 2, 3, 4\}  \cap \{2, 3, 4, 5\}$

        \textcolor{hwColor}{
          \\
          $
            \{2, 3, 4\} ~~~~ \checkmark
          $
          \\
        }

    \end{enumerate}

    \item Write each of the following in interval notation:
    \begin{enumerate}
      \item $(0, 2) \cap (\dfrac{1}{2}, 1)$

        \textcolor{hwColor}{
          \\
          $
            (0, 2) \cap (\dfrac{1}{2}, 1)=\left(\dfrac{1}{2}, 1\right) ~~~~ \checkmark
          $
          \\
        }

      \item $[-1, 5] \cup [2, 7]$

        \textcolor{hwColor}{
          \\
          $
            [-1, 7] ~~~~ \checkmark
          $
          \\
        }

    \end{enumerate}

    \item Prove (vi) of Theorem 0.2
    
      \textcolor{hwColor}{
        We are asked to prove $(A \cup B) \cup C=A \cup (B \cup C)$. Let $X=((A \cup B) \cup C)$ and $Y=A \cup (B \cup C)$. 
        \\
        \\
        If $x \in Y$, it means $x \in (A \cup B)$, therefore $x \in A$ or $x \in B$ or $x \in C$ so $X \subset Y$.
        \\
        \\
        If $x \in X$, it means $x \in (B \cup C)$, therefore $x \in A$ or $x \in B$ or $x \in C$ so $Y \subset X$.
        \\
        \\
        \\
        $\therefore ~~~~ Y \subset X$ and $X \subset Y$, hence $Y=X ~~~~~ \checkmark$ 
      }

  \end{enumerate}

  \rule{15cm}{2pt}

  \textbf{0.2}
  \begin{enumerate}
    \item Give an example of $f: A \rightarrow B$ that is not $1-1$.

      \textcolor{hwColor}{
        $f$ is one-to-one (injective) if $f$ maps every element of $A$ to a unique element in $B$. 
        In other words no element of $B$ are mapped to by two or more elements of $A$.
        \\
        \\
        $
          \left( \forall a,b \in A \right) ~ f(a)=f(b) \Rightarrow a=b.
        $
        \\
        \\
        Suppose we have $A=\{-1,0\}$ and $B=\{1, 2\}$. And suppose there is a function $f$
        such that $f: A \rightarrow B$ so we have $\{ (-1,1), (0,1)\}$. Based on the 
        definition of one-to-one functions, the function is not 1-1.
        \\
      }

    \item If $f: A \rightarrow B$ is $1-1$ and $im f=B$, prove that $(f^{-1}of)(a)=a$
    for all $a \in A$ and $(fof^{-1})(b)=b$ for each $b \in B$.

      \textcolor{hwColor}{
        Range is sometimes used to mean the same as "codomain", and sometimes to mean the subset of the codomain whose members are "used", 
        that is, for which an element of the domain exists which the function maps to that element of the codomain. It’s the image of the domain.
        \\
        \\
        Image is usually used of specific subsets of the domain; the image of a subset  S  of the domain is the set $\{f(s): s \in S \}$. 
        It’s the subset of the codomain for which at least one element of  S  is mapped to each of its elements. Sometimes, 
        when no particular subset of the domain is explicitly specified, the entire domain is meant; the same as the more 
        restricted meaning of "range", not identical by definition to the codomain.
        \\
        \\
        Suppose $x \in A$ and $y \in B$. With the given conditions:
        \\
        \\
        $f: A \rightarrow B$ is bijection which suggests that $f^{-1}: B \rightarrow A$ is one-to-one. 
        \\
        \\
        Hence, \begin{itemize}
          \item $\left(f f^{-1}\right)$ maps to $B \rightarrow B$
          \item $\left(f^{-1} f\right)$ maps to $A \rightarrow A$
        \end{itemize}
      }

      \textcolor{hwColor}{
        $
          \begin{cases}
            (f^{-1}of)(a)=a
            \\
            \\
            (fof^{-1})(a)=b
          \end{cases} ~~~~ \checkmark
        $ 
      }

  \end{enumerate}

  \rule{15cm}{2pt}

  \textbf{0.3}
  \begin{enumerate}
    \item Prove that for all $n \in \mathcal{J}, 1+2+...+n=\dfrac{n(n+1)}{2}$.

      \textcolor{hwColor}{
        \textbf{Basis: $n=1$}
        \\
        \\
        $
          1=1 \Longrightarrow L.H.S=R.H.S ~~~~ \checkmark 
        $
        \\
        \\
        \textbf{Induction:}
        \\
        \\
        Assume $1+2+...+n=\dfrac{n(n+1)}{2}$ is true for some $n=k$. Therefore, $1+2+...+k=\dfrac{k(k+1)}{2}$.
        \\
        \\
        Now we need to show that the statement is also true for $n=k+1$.
        \\
        \\
        $
          1+2+...+k+(k+1)=\dfrac{(k+1)\left[(k+1)+1\right]}{2}
          \\
          \\
          \\
          \therefore ~~~~ \dfrac{k(k+1)}{2}+(k+1)=\dfrac{(k+1)(k+2)}{2}
          \\
          \\
          \\
          \therefore ~~~~ \dfrac{k(k+1)}{2}+\dfrac{2(k+1)}{2}=\dfrac{(k+1)(k+2)}{2}
          \\
          \\
          \\
          \therefore ~~~~ k(k+1)+2(k+1)=(k+1)(k+2)
          \\
          \\
          \\
          \therefore ~~~~ k^2+k+2k+2=k^2+2k+k+2 ~~~~ \checkmark
        $
        \\
        \\
        So we basically showed that this is true for $n=k+1$.
      }

    \item Prove that for all $n \in \mathcal{J}, 1+3+5+...+(2n-1)=n^2$.
    
      \textcolor{hwColor}{
        \textbf{Basis: $n=1$}
        \\
        \\
        $
          1=1 \Longrightarrow L.H.S=R.H.S ~~~~ \checkmark 
        $
        \\
        \\
        \textbf{Induction:}
        \\
        \\
        Assume $1+3+5+...+(2n-1)=n^2$ is true for some $n=k$. 
        \\
        \\
        Therefore, $1+3+5+...+(2k-1)=k^2 ~~~~~ (A)$
        \\
        \\
        Now we need to show that the statement is also true for $n=k+1$.
        \\
        \\
        $
          1+3+5+...+ \left[2(k+1)-1\right]=(k+1)^2
          \\
          \\
          \\
          \therefore ~~~~ 1+3+5+...+\left(2k-1\right)+\left(2k+1\right)=(k+1)^2
          \\
          \\
          \\
        $
        From $(A)$, we know $1+3+5+...+(2k-1)=k^2$, hence:
        \\
        \\
        $
          \therefore ~~~~ k^2+\left(2k+1\right)=(k+1)^2
          \\
          \\
          \\
          \therefore ~~~~ \left(k+1\right)^2=\left(k+1\right)^2 ~~~~ \checkmark
        $
        \\
        \\
        So by induction, for every positive integer $\mathcal{J}$ this $1+3+5+...+(2n-1)=n^2$ holds.
      }

    \item Define $f(n)$ as follows for $n \in \mathbb{Z}, n\geqslant 0$. $f(0)=7, f(1)=4$
    and, for $n \geqslant 2, f(n)=6f(n-2)-f(n-1)$. Prove that $f(n)=5(2)^n+2(-3)^n$ for 
    all $n \in \mathbb{Z}, n \geqslant 0$.

      \textcolor{hwColor}{
        $
          f(0)=5(2^0)+2(-3)^0=7
          \\
          \\
          \\
          f(1)=5(2)^1-2^1+2(-3)^1=4
          \\
          \\
          \\
          f(k-1)=5(2)^{k-1}+2(-3)^{k-1}
          \\
          \\
          \\
          f(k+1)=6 \left[f(k+1-2)\right]-f(k+1-1)
          \\
          \\
          =6\left[5(2)^{k-1}+2(-3)^{k-1}\right]-\left[5(2)^k+2(-3)^k\right]
          \\
          \\
          \\
          \therefore f(k+1)=6 \left[f(k+1-2)\right]-f(k+1-1)=5(2)^{k+1}-2(-3)^{k+1} ~~~~ \checkmark
        $
        \\
        \\
        \\
        So by induction, $f(n)=5(2)^n+2(-3)^n$ holds.
      }

  \end{enumerate}

  \rule{15cm}{2pt}

  \textbf{0.5}
  \begin{enumerate}
    \item If $x < y$, prove that $x < \dfrac{x+y}{2} < y$.

      \textcolor{hwColor}{
        $
          x<y \longrightarrow \dfrac{x}{2} < \dfrac{y}{2}
          \\
          \\
          \\
          \therefore ~~~~ x<\dfrac{x}{2}+\dfrac{y}{2}<\dfrac{x+y}{2}
          \\
          \\
          \\
          \therefore ~~~~ \dfrac{x}{2}+\dfrac{y}{2}=\dfrac{x+y}{2}<\dfrac{y}{2}+\dfrac{y}{2}=y
          \\
          \\
          \\
          \therefore ~~~~ x < \dfrac{x+y}{2} < y ~~~~ \checkmark
        $
      }

    \item If $x \geq 0$ and $y \geq 0$, prove that $\sqrt{xy} \leq \dfrac{x+y}{2}$. 
    [Hint: Use the fact that $\left(\sqrt{x}-\sqrt{y}\right)^2 \geqslant 0$.]

      \textcolor{hwColor}{
        We know that $\left(\sqrt{x}+\sqrt{y}\right)^2 \geqslant 0$ then:
        \\
        \\
        $
          x+y \geqslant 2\sqrt{xy}
          \\
          \\
          \\
          \therefore ~~~~ \sqrt{xy} \leq \dfrac{x+y}{2} ~~~~ \checkmark
        $
      }

    \item If $0 < a < b$, prove that $0 < a^2 < b^2$ and $0 < \sqrt{a} < \sqrt{b}$.

      \textcolor{hwColor}{
        The following can be written based on the given context:
        \\
        \\
        \textbf{A:}
        \\
        \\
        $
          \begin{cases}
            0<a^2<ab
            \\
            \\
            0<ab<b^2  
          \end{cases} \Longrightarrow 0<a^2<ab<b^2 \Longrightarrow 0<a^2<b^2 ~~~~ \checkmark
        $
        \\
        \\
        \\
        \textbf{B:}
        \\
        \\
        $
          0<a<b \Longrightarrow a-b<0 \Longrightarrow \left(\sqrt{a}-\sqrt{b}\right)\left(\sqrt{a}+\sqrt{b}\right)<0
          \\
          \\
          \\
          \dfrac{\left(\sqrt{a}-\sqrt{b}\right)\left(\sqrt{a}+\sqrt{b}\right)}{\left(\sqrt{a}+\sqrt{b}\right)}<0
          \\
          \\
          \\
          \therefore ~~~~ \sqrt{a}-\sqrt{b}<0
          \\
          \\
          \\
          \therefore ~~~~ \sqrt{a}<\sqrt{b} ~~~~ \checkmark
        $
      }

    \item If $x=sup ~ S$, show that, for each $\epsilon > 0$, there is $a \in S$ such that
    $x-\epsilon < a \leq x$.

      \textcolor{hwColor}{
        We are told that $x$ is upper bound of $S$. and we know that $x \geq a \in S$ and $x-\epsilon \in S$.
        $
          \\
          \\
          \\
          \therefore ~~~~ a>x-\epsilon
          \\
          \\
          \\
          \therefore ~~~~ x-\epsilon <a \leq x ~~~~ \checkmark
        $
      }

  \end{enumerate}

  \rule{15cm}{2pt}

  \textbf{1.1}
  \begin{enumerate}
    \item Show that $[0, 1]$ is a neighborhood of $\dfrac{2}{3}$ that is, there is $\epsilon > 0$ such that
    $$
      \left(\dfrac{3}{2}-\epsilon, \dfrac{3}{2}+\epsilon\right) \subset [0,1]
    $$

      \textcolor{hwColor}{
        Assume $\epsilon=\dfrac{1}{3}$, then we have:
        \\
        \\
        $
          \left[\dfrac{2}{3}-\epsilon, \dfrac{2}{3}+\epsilon\right]=\left(\dfrac{1}{3}, 1\right)=\left[\dfrac{2}{3}-\dfrac{1}{3}, \dfrac{2}{3}+\dfrac{1}{3}\right]
        $
        \\
        \\
        \\
        $
          \left(\dfrac{1}{3}, 1\right) \subset \left[0,1\right]
        $ is a neighborhood for $\dfrac{2}{3} ~~~~ \checkmark$  
      }

    \item Find upper and lower bounds for the sequence $\{\dfrac{3n+7}{n}\}_{n=1}^{\infty}$.

      \textcolor{hwColor}{
        \textbf{The upper bound:}
        \\
        \\
        $
          n \geq 1\longrightarrow 7n \geq 7 
        $
        \\
        \\
        Therefore $10n \geq 3n+7$ suggests $10 \geq \dfrac{3n+7}{n} ~~~~ \checkmark$
        \\
        \\
        \\
        \textbf{The lower bound:}
        \\
        \\
        $7+3n >3n$ suggests $3<\dfrac{3n+7}{n}$. Therefore, $3<\{\dfrac{3n+7}{n}\}_{n=1}^{\infty} \leq 10 ~~~~ \checkmark$
      }

    \item Give an example of a sequence that is bounded but not convergent.
    
      \textcolor{hwColor}{
        $(-1)^n cos(n), ~~ n \in \mathcal{N}$ is a bounded sequence, its lower bound is $-1$ and upper 
        bound is $1$. Since it oscillates between $1$ and $-1$ it is not convergent. $~~~ \checkmark$
      }

    \item Use the definition of convergence to prove that each of the following sequences converges:
    \begin{enumerate}
      \item $\{5+\dfrac{1}{n}\}_{n=1}^{\infty}$.

        \textcolor{hwColor}{
          Let $\epsilon > 0$ and let's assume $K>\dfrac{1}{\epsilon}$. Then for any number less than $K$ we have:
          \\
          \\
          $
            |a_n-a|=|5+\dfrac{1}{n}-5|=\dfrac{1}{n} \leq \dfrac{1}{K} < \epsilon 
            \\
            \\
            \\
            \therefore ~~~~ a=5 ~~~~ \checkmark
          $
          \\
          \\
        }
      
      \item $\{\dfrac{2-2n}{n}\}_{n=1}^{\infty}$.

        \textcolor{hwColor}{
          Let $\epsilon > 0$ and let's assume $K>\dfrac{2}{\epsilon}$.
          \\
          \\
          $
            |a_n-a|=|\dfrac{2-2n}{n}-a|=|\dfrac{2}{n}-2+2|=\dfrac{2}{n} \leq \dfrac{2}{K}< \epsilon
            \\
            \\
            \\
            \therefore ~~~~ a=-2 ~~~~ \checkmark
          $
          \\
        }

      \item $\{2^{-n}\}_{n=1}^{\infty}$.

        \textcolor{hwColor}{
          Let $\epsilon > 0$ and let's assume $K>\dfrac{ln(\dfrac{1}{\epsilon})}{ln(2)}$.
          \\
          \\
          $
            |a_n-a|=|2^{-n}-0|=2^{-n} <\epsilon
            \\
            \\
            \\
            \therefore ~~~~ a=0 ~~~~ \checkmark
          $
          \\
        }

      \item $\{\dfrac{3n}{2n+1}\}_{n=1}^{\infty}$.

        \textcolor{hwColor}{
          let $\epsilon > 0$ and let's assume $K>\dfrac{3}{4 \epsilon}$.
          \\
          \\
          $
            |a_n-a|=|\dfrac{3n}{2n+1}-\dfrac{3}{2}|=|-\dfrac{3}{4n+2}|=\dfrac{3}{4n+2}<\dfrac{3}{4n}\leq \dfrac{3}{4K}
            \\
            \\
            \\
            \therefore ~~~~ a=\dfrac{3}{4} ~~~~ \checkmark
          $
        }

    \end{enumerate}

    \item Suppose $\{a_n\}_{n=1}^{\infty}$ converges to $A$, and define a new sequence $\{b_n\}_{n=1}^{\infty}$ by
    $b_n=\dfrac{a_n+a_{n+1}}{2}$ for all $n$. Prove that $\{b_n\}_{n=1}^{\infty}$ converges to $A$.

      \textcolor{hwColor}{
        Let $\epsilon > 0$. We are told that this sequence converges to $A$.
        \\
        \\
        $
          |b_n-A|=|\dfrac{a_n+a_{n+1}}{2}-A|=\dfrac{1}{2}|a_n+a_{n+1}-2A|=\dfrac{1}{2}|(a_n-A)+(a_{n+1}-A)|\leq \dfrac{1}{2} ||+\dfrac{1}{2} |a_{n+1}-A| < \epsilon
        $
        \\
        \\
        \\
        Hence, it converges to $A$. 
      }

  \end{enumerate}

  \rule{15cm}{2pt}

  \textbf{1.2}
  \begin{enumerate}
    \item Prove that every Cauchy sequence is bounded (Theorem 1.4).

      \textcolor{hwColor}{
        Let $\epsilon=1$. Because this sequence is Cauchy, there exists some positive integer $N$ such that
        \\
        \\
        $
          |a_n-a_m|<1 \therefore \begin{cases}
            a_m-1<a_n<a_m+1
            \\
            \\
            a_n < a_m+1
          \end{cases}
          \\
          \\
          \\
        $
        Therefore we have our supremum as:
        \\
        \\
        $
          \therefore ~~~~ sup \{|a_1|, |a_2|, ..., |a_N|,|a_N|+1\}
        $
        \\
        \\
        We we have have now is bound by $\{a_n\}_{n=1}^{\infty}$. Hence every Cauchy sequence is bounded $ ~~~~ \checkmark$ 
        \\
      }

    \item Prove that the sequence $\{\dfrac{2n+1}{n}\}_{n=1}^{\infty}$ is Cauchy.

      \textcolor{hwColor}{
        Let $\epsilon > 0$. Then there must be an integer $N$ such that $\dfrac{1}{N}<\dfrac{\epsilon}{2}$.
        \\
        \\
        $
          |a_n-a_z|=|\dfrac{2n+1}{n}-\dfrac{2z+1}{m}|=|\dfrac{1}{n}-\dfrac{1}{z}|=\dfrac{1}{n}-\dfrac{1}{z} < \epsilon
          \\
          \\
        $
        Hence, the sequence $\{\dfrac{2n+1}{n}\}_{n=1}^{\infty}$ is Cauchy. $~~~ \checkmark$
        \\
      }

    \item Give an example of a set with exactly two accumulation points.

      \textcolor{hwColor}{
        Set $A=\{(-1)^{n+1} | n \in \mathcal{N} \}$. This has only $1$ and $-1$ as its accumulation points.
        \\
        \\
        \\
        Set $A=\{\dfrac{n(-1)^n}{n+1} | n \in \mathcal{N} \}$. This has only $1$ and $-1$ as its accumulation points.
        \\
      }

    \item Let $a_0$ and $a_1$ be distinct real numbers. Define $a_n=\dfrac{a_{n-1}+a_{n-2}}{2}$ for each positive integer
    $n \geqslant 2$. Show that $\{a_n\}_{n=1}^{\infty}$ is a Cauchy sequence. You maywant to use induction to show that
    $$
      a_{n+1}-a_n=\left(-\dfrac{1}{2}\right)^n \left(a_1-a_0\right)
    $$
    and then use the result from Example 0.9 of Chapter 0.

      \textcolor{hwColor}{
        \\
        \\
        \textbf{Basis: $n=0$}
        \\
        \\
        $
          a_{1}-a_0=\left(-\dfrac{1}{2}\right)^0 \left(a_1-a_0\right)=\left(a_1-a_0\right)
        $
        \\
        \\
        The formula holds so far.
        \\
        \\
        \textbf{Induction: $n=N+1$}
        \\
        \\
        For the left-hand-side we have:
        \\
        \\
        $
          a_{n+2}-a_{N+1}=-\left(\dfrac{1}{2}\right)\left(a_{N+1}-a_N\right)=-\left(\dfrac{1}{2}\right)^{N+1} \left(a_1-a_0\right)
        $
        \\
        \\
        \\
        What we just found the equal to the right-hand-side. Therefore $a_{n+1}-a_n=(1\dfrac{1}{2})(a_1-a_0)$ holds for all values of $n \in \mathcal{Z}$.
        \\
        \\
        Now let's have $\epsilon>0$ and $n>z\geq N \in \mathcal{Z}$. We have the following:
        \\
        \\
        $
          |a_n-a_z|=|a_n-a_{n-1}+a_{n-1}+...+a_{z+1}-a_z| \leq |a_n-a_{n-1}|+|a_{n-1}-a_{n-2}|+...+|a_{z+1}-a_z|
          \\
          \\
          \\
          =|-\left(\dfrac{1}{2}\right)^{n-1} (a_1-a_0)|+|-\left(\dfrac{1}{2}\right)^{n-2} (a_1-a_0)|+...+|-\left(\dfrac{1}{2}\right)^z (a_1-a_0)|
          \\
          \\
          \\
          =\left[-\left(\dfrac{1}{2}\right)^{n-1}+-\left(\dfrac{1}{2}\right)^{n-2}+...+-\left(\dfrac{1}{2}\right)^z\right]|a_1-a_0| <-\left(\dfrac{1}{2}\right)^{z-1} |a_1-a_0| < \epsilon ~~~~ \checkmark
        $
        \\
        \\
        Hence, for all $n > z \geq N$, the above holds. Therefore, the sequence is Cauchy.
        \\
      }

  \end{enumerate}

  \rule{15cm}{2pt}

  \textbf{1.3}
  \begin{enumerate}
    \item Suppose $\{a_n\}_{n=1}^{\infty}$ and $\{b_n\}_{n=1}^{\infty}$ are sequences such that $\{a_n\}_{n=1}^{\infty}$
    and $\{a_n+b_n\}_{n=1}^{\infty}$ converge. Prove that $\{b_n\}_{n=1}^{\infty}$ converges.

    \item Give an example in which $\{a_n\}_{n=1}^{\infty}$ and $\{b_n\}_{n=1}^{\infty}$ l do not converge but 
    $\{a_n+b_n\}_{n=1}^{\infty}$ converges.


    \item Find the limit of the sequences with general term as given:
    \begin{enumerate}
      \item $\dfrac{n^2+4n}{n^2-5}$.

      \item $\dfrac{cos(n)}{n}$.
      
      \item $\dfrac{sin(n^2)}{\sqrt{n}}$.
      
      \item $\dfrac{n}{n^2-3}$.
      
      \item $\left(\sqrt{4-\dfrac{1}{n}}-2\right)n$.

      \item $(-1)^n \dfrac{\sqrt{n}}{n+7}$.
    \end{enumerate}

  \end{enumerate}

\end{document}
