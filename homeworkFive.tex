\documentclass[fleqn]{article}
\oddsidemargin 0.0in
\textwidth 6.0in
\thispagestyle{empty}
\usepackage{import}
\usepackage{amsmath}
\usepackage{graphicx}
\usepackage{flexisym}
\usepackage{amssymb}
\usepackage{bigints} 
\usepackage[english]{babel}
\usepackage[utf8x]{inputenc}
\usepackage{float}
\usepackage[colorinlistoftodos]{todonotes}

\definecolor{hwColor}{HTML}{042c4d}

\begin{document}

  \begin{titlepage}

    \newcommand{\HRule}{\rule{\linewidth}{0.5mm}}

    \center


    \textsc{\LARGE Arizona State University}\\[1.5cm]

    \textsc{\LARGE Advanced Calculus I }\\[1.5cm]


    \begin{figure}
      \includegraphics[width=\linewidth]{asu.png}
    \end{figure}


    \HRule \\[0.4cm]
    { \huge \bfseries Homework Five }\\[0.4cm] 
    \HRule \\[1.5cm]

    \textbf{Behnam Amiri}

    \bigbreak

    \textbf{Prof: Sergei Suslov}

    \bigbreak


    \textbf{{\large \today}\\[2cm]}

    \vfill

  \end{titlepage}

  \textbf{5.3 RIEMANN SUMS}
  \begin{enumerate}
    \item (14) Suppose $f$ is integrable on $[-b, b]$ and $f$ is an odd function that is $f(-t)=-f(t)$ for all
    $t \in [-b, b]$. Prove that $\bigints\limits_{-b}^{+b} f dx=0$. If $f$ is even that is $f(-t)=f(t)$ for all 
    $t \in [-b, b]$, prove that 
    $$\bigints\limits_{-b}^{+b} f dx=2 \bigints\limits_{0}^{+b} f dx$$

      \textcolor{hwColor}{
        Since we know the integral is additive with respect to the interval of integration, we have:
        \\
        \\
        $
          \bigints\limits_{-b}^{+b} f(x) dx=\bigints\limits_{-b}^{0} f(x) dx+\bigints\limits_{0}^{+b} f(x) dx  
        $
        \\
        \\
        Then, for the first integral we use the expansion/contraction of the interval of integration with $k=-1$ to get
        \\
        \\
        $
          \bigints\limits_{-b}^{0} f(x) dx=-\bigints\limits_{+b}^{0} f(-x) dx
        $
        \\
        \\
        Since $f(x)$ is an even function by assumption, we have $f(-x)=f(x)$ for all $x \in [o, b]$. And since 
        $-\bigints\limits_{b}^{0}=\bigints\limits_{0}^{b}$ we then have,
        \\
        \\
        $
          -\bigints\limits_{+b}^{0} f(-x) dx=\bigints\limits_{0}^{+b} f(x) dx
        $
        \\
        \\
        So, putting this all together we have,
        \\
        \\
        $
          \bigints\limits_{-b}^{+b} f(x) dx
          =\bigints\limits_{0}^{+b} f(x)dx+\bigints\limits_{0}^{+b}
          \\
          \\
          \\
          \therefore ~~~~ \bigints\limits_{-b}^{+b} f(x) dx=2\bigints\limits_{0}^{+b} f(x) dx ~~~~~ \blacksquare 
        $
        \\
        \\
        \\
        \\
        \\
        \\
        Similarly we have,
        \\
        \\
        $
          \bigints\limits_{-b}^{+b} f(x) dx
          =\bigints\limits_{-b}^{0} f(x) dx+\bigints\limits_{0}^{+b} f(x)dx
          =\bigints\limits_{0}^{+b} f(-x) dx+\bigints\limits_{0}^{+b} f(x) dx
        $
        \\
        \\
        Since $f$ is an odd function, we have $f(-x)=-f(x)$ for all $x \in [0, b]$. Thus,
        \\
        \\
        $
          \bigints\limits_{0}^{+b} f(-x) dx
          =-\bigints\limits_{0}^{+b} f(x) dx
          \\
          \\
          \\
          \therefore ~~~~ \bigints\limits_{-b}^{+b} f(x) dx=0 ~~~~~ \blacksquare
        $
        \\
        \\
      }

    \item (15) Suppose that $f: R \longrightarrow R$ is periodic and integrable on every closed interval. If $p$ is the
    period of $f$, prove that for any $a \in R$,
    $$\bigints\limits_{0}^{p} f dx=\bigints\limits_{a}^{a+p} f dx$$

      % \textcolor{hwColor}{
        
      % }

  \end{enumerate}


  \textbf{5.4 THE FUNDAMENTAL THEOREM OF INTEGRAL CALCULUS}
  \begin{enumerate}
    \item (16) Use the Fundamental Theorem of Integral Calculus to compute the following:
    \begin{enumerate}
      \item $\bigints\limits_{0}^{3} \left(x^2-x\right) dx$

        % \textcolor{hwColor}{
          
        % }

      \item $\bigints\limits_{-2}^{4} \left(1-x^3-x^2\right) dx$

        % \textcolor{hwColor}{
          
        % }

      \item $\bigints\limits_{0}^{\dfrac{\pi}{2}} x sin(x^2)$

        % \textcolor{hwColor}{
          
        % }

    \end{enumerate}

    \item (17) Define $f: [0, 2] \longrightarrow R$ by $f(x)=2x-x^2$ for $0 \leq x \leq 1$ and $f(x)=(x-2)^2$
    for $1 \leq x \leq 2$. Prove that $f$ is integrable on $[0, 2]$ and find the integral of $f$ over $[0, 2]$.
    Do not use Theorem $5.10$, but rather find the integral by methods similar to those used in the proof
    of Theorem $5.8$.

      % \textcolor{hwColor}{
          
      % }

  \end{enumerate}

  \textbf{5.5 ALGEBRA OF INTEGRABLE FUNCTIONS}
  \begin{enumerate}
    \item (18) Suppose $f$ and $g$ are differentiable on $[a, b]$ and $f^'$ and $g^'$ are integrable on $[a, b]$.
    Prove that $f^' g$ and $g^' f$ are integrable on $[a, b]$ and that
    $$\bigints\limits_{a}^{b} f^' g dx=f(b) g(b)-f(a) g(a)-\bigints\limits_{a}^{b} g^' f dx$$
    Of course, this is the \emph{integration-by-parts} formula.

      % \textcolor{hwColor}{
          
      % }

  \end{enumerate}


  \textbf{5.6 DERIVATIVES OF INTEGRALS}
  \begin{enumerate}
    \item (30) Find $f^'$ where $f$ is defined on $[0, 1]$ as indicated:
    \begin{enumerate}
      \item $f(x)=\bigints\limits_{0}^{x} \sqrt{t^2+1} dt$

        % \textcolor{hwColor}{
          
        % }

      \item $f(x)=\bigints\limits_{x}^{1} cos\left(\dfrac{1}{t+1}\right) dt$

        % \textcolor{hwColor}{
          
        % }

      \item $f(x)=\bigints\limits_{x^2}^{2x} sin(t^2) dt$

        % \textcolor{hwColor}{
          
        % }

      \item $f(x)=\bigints\limits_{x}^{\sqrt{x}} \dfrac{1}{1+t^3} dt$

        % \textcolor{hwColor}{
          
        % }

    \end{enumerate} 

  \end{enumerate}


  \textbf{5.7 MEAN-VALUE AND CHANGE-OF-VARIABLE THEOREMS}
  \begin{enumerate}
    \item (33) Suppose $f: R \longrightarrow R$ is continuous and has period $p$, so that $f(x+p)=f(x)$ for all $x \in R$.
    Show that $\bigints\limits_{x}^{x+p} f(t) dt$ is independent of $x$ in that, for all $x, y,$
    $$\bigints\limits_{x}^{x+p} f(t) dt=\bigints\limits_{y}^{y+p} f(t) dt$$
    Show, then, that $\bigints\limits_{0}^{p}  \left[f(x+a)-f(x)\right]dx=0$ for any real number $a$. Conclude that for any
    real number $a$, there is $x$ such that $f(x+a)=f(x)$.

      % \textcolor{hwColor}{
          
      % }

    \item (35) Use Theorem $5.18$ to evaluate the integrals:
    \begin{enumerate}
      \item $\bigints\limits_{0}^{3} \sqrt[3]{1+x^2} xdx$

        % \textcolor{hwColor}{
          
        % }

      \item $\bigints\limits_{1}^{4} \dfrac{\left(\sqrt{x}+2\right)^3}{\sqrt{x}} dx$

        % \textcolor{hwColor}{
          
        % }

      \item $\bigints\limits_{1}^{\sqrt{3}} \dfrac{\sqrt{x^2-9}}{x} dx$

        % \textcolor{hwColor}{
          
        % }

      \item $\bigints\limits_{0}^{1} \dfrac{x^2}{\sqrt{1-x^2}} dx$

        % \textcolor{hwColor}{
          
        % }

    \end{enumerate}
    [You may want to consult your calculus book before trying (c) and (d).] 


  \end{enumerate}

  \textbf{6.1 CONVERGENCE OF INFINITE SERIES}
  \begin{enumerate}
    \item (1) Let $\{ a_n \}_{n=1}^{\infty}$ be a sequence of real numbers. Prove that
    $$\sum\limits_{n=1}^{\infty} (a_n-a_{n+1})$$
    converges iff $\{ a_n \}_{n=1}^{\infty}$ converges. If $\sum\limits_{n=1}^{\infty} (a_n-a_{n+1})$ converges, 
    what is the sum?

      % \textcolor{hwColor}{
          
      % }

    \item (3) Prove that $\sum\limits_{n=1}^{\infty} 2^n r^n$ converges if $|r| < \dfrac{1}{2}$ and find the sum. 

      % \textcolor{hwColor}{
          
      % }

    \item (4) Prove that the series $\sum\limits_{n=0}^{\infty} 3^{-n}$ converges and find the limit. 

      % \textcolor{hwColor}{
          
      % }

    \item (6) Use induction to show that $1+\dfrac{1}{\sqrt{2}}+...+\dfrac{1}{\sqrt{n}} \geq \sqrt{n}$ for $n \geq 1$.
    Can you use this fact to determine whether the series $\sum\limits_{n=1}^{\infty} \dfrac{1}{\sqrt{n}}$ converges or diverges? 
 
      % \textcolor{hwColor}{
          
      % }

  \end{enumerate}


  \textbf{6.2 ABSOLUTE CONVERGENCE AND THE COMPMISON TEST}
  \begin{enumerate}
    \item (13) Suppose $\sum\limits_{n=1}^{\infty} a_n$ converges absolutely and $\{ b_n \}_{n=1}^{\infty}$ is bounded.
    Prove that $\sum\limits_{n=1}^{\infty} a_n b_n$ converges absolutely.

      % \textcolor{hwColor}{
          
      % }

    \item (15) Determine which of the following infinite series converge.
    \begin{enumerate}
      \item $\sum\limits_{n=1}^{\infty} \dfrac{1}{(n+1)(2n-1)}$

        % \textcolor{hwColor}{
          
        % }

      \item $\sum\limits_{n=1}^{\infty} \dfrac{n}{n+1}$

        % \textcolor{hwColor}{
          
        % }

      \item $\sum\limits_{k=1}^{\infty} \dfrac{2}{3k}$
      
        % \textcolor{hwColor}{
          
        % }

      \item $\sum\limits_{m=1}^{\infty} \dfrac{\sqrt{m+1}-\sqrt{m}}{m}$

        % \textcolor{hwColor}{
          
        % }

      \item $\sum\limits_{m=1}^{\infty} \dfrac{3^m}{5^{m+1}}$

        % \textcolor{hwColor}{
          
        % }

      \item $\sum\limits_{m=1}^{\infty} \dfrac{1}{m^2} \left(\dfrac{100m+1}{m}\right)$

        % \textcolor{hwColor}{
          
        % }

    \end{enumerate}

    \item (16) (Limit-comparison test.) Prove the following generalization of Theorem $6.5$. Suppose
    $\sum\limits_{n=0}^{\infty} a_n$ and $\sum\limits_{n=0}^{\infty} b_n$ are series of positive terms such that 
    $$\{ \dfrac{a_n}{b_n} \}_{n=1}^{\infty}$$
    converges to $L\neq 0$. Then $\sum\limits_{n=0}^{\infty} a_n$ and $\sum\limits_{n=0}^{\infty} b_n$  either both diverge or both converge.
    What can be concluded if $L=0$

      % \textcolor{hwColor}{
          
      % }
 
  \end{enumerate}


  \textbf{6.3 RATIO AND ROOT TESTS}
  \begin{enumerate}
    \item (19) Use Theorem 6.9 to determine the values of $r$ for which $\sum\limits_{n=0}^{\infty} n r^n$ converges.

      % \textcolor{hwColor}{
          
      % }

    \item (20) Prove that $\{ n x^n\}_{n=1}^{\infty}$ converges to zero if $|x| < 1$.

      % \textcolor{hwColor}{
          
      % }

  \end{enumerate}


  \textbf{6.5 POWER SERIES}
  \begin{enumerate}
    \item (34) Show that $\sum\limits_{n=1}^{\infty} n! x^n$ converges only for $x=0$.

      % \textcolor{hwColor}{
          
      % }

    \item (35) Show that $\sum\limits_{n=1}^{\infty} \dfrac{x^n}{n}$ converges iff $-1 \leq x <1$.

      % \textcolor{hwColor}{
          
      % }

    \item (38) Show that the power series $\sum\limits_{n=0}^{\infty} a_n x^n$ and $\sum\limits_{n=1}^{\infty} n a_n x^{n-1}$ either both converge for all 
    $x$, both converge only for $x=0$, or both have the same finite nonzero radius of convergence.

      % \textcolor{hwColor}{
          
      % }

  \end{enumerate}


  \textbf{6.6 TAYLOR SERIES}
  \begin{enumerate}
    \item (42) Write the Taylor series for $log ~ x$ using powers $x-1$. See project $5.1$ in Chapter 5 if
    you have forgotten the derivative of $log ~ x$. Prove that this series converges to $log ~ x$ for $1 \leq x <2$.

      % \textcolor{hwColor}{
          
      % }

    \item (43) Write the Taylor series for $\sqrt{1-x}$ in power of $x$. Prove that the series converges to $\sqrt{1-x}$ for
    $-1 < x \leq 0$.

      % \textcolor{hwColor}{
          
      % }

    \item (44) If $x > 0$, show that $|\left(1+x\right)^{\dfrac{1}{3}}-\left(1+\dfrac{x}{3}-\dfrac{x^2}{9}\right)| \leq \dfrac{5x^3}{81}$.

      % \textcolor{hwColor}{
          
      % }

  \end{enumerate}

\end{document}
