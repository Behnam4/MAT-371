\documentclass[fleqn]{article}
\oddsidemargin 0.0in
\textwidth 6.0in
\thispagestyle{empty}
\usepackage{import}
\usepackage{amsmath}
\usepackage{graphicx}
\usepackage{flexisym}
\usepackage{amssymb}
\usepackage{bigints} 
\usepackage[english]{babel}
\usepackage[utf8x]{inputenc}
\usepackage{float}
\usepackage[colorinlistoftodos]{todonotes}

\definecolor{hwColor}{HTML}{042c4d}

\begin{document}

  \begin{titlepage}

    \newcommand{\HRule}{\rule{\linewidth}{0.5mm}}

    \center


    \textsc{\LARGE Arizona State University}\\[1.5cm]

    \textsc{\LARGE Advanced Calculus I }\\[1.5cm]


    \begin{figure}
      \includegraphics[width=\linewidth]{asu.png}
    \end{figure}


    \HRule \\[0.4cm]
    { \huge \bfseries Homework Five }\\[0.4cm] 
    \HRule \\[1.5cm]

    \textbf{Behnam Amiri}

    \bigbreak

    \textbf{Prof: Sergei Suslov}

    \bigbreak


    \textbf{{\large \today}\\[2cm]}

    \vfill

  \end{titlepage}

  \textbf{5.3 RIEMANN SUMS}
  \begin{enumerate}
    \item (14) Suppose $f$ is integrable on $[-b, b]$ and $f$ is an odd function that is $f(-t)=-f(t)$ for all
    $t \in [-b, b]$. Prove that $\bigints\limits_{-b}^{+b} f dx=0$. If $f$ is even that is $f(-t)=f(t)$ for all 
    $t \in [-b, b]$, prove that 
    $$\bigints\limits_{-b}^{+b} f dx=2 \bigints\limits_{0}^{+b} f dx$$

      \textcolor{hwColor}{
        Since we know the integral is additive with respect to the interval of integration, we have:
        \\
        \\
        $
          \bigints\limits_{-b}^{+b} f(x) dx=\bigints\limits_{-b}^{0} f(x) dx+\bigints\limits_{0}^{+b} f(x) dx  
        $
        \\
        \\
        Then, for the first integral we use the expansion/contraction of the interval of integration with $k=-1$ to get
        \\
        \\
        $
          \bigints\limits_{-b}^{0} f(x) dx=-\bigints\limits_{+b}^{0} f(-x) dx
        $
        \\
        \\
        Since $f(x)$ is an even function by assumption, we have $f(-x)=f(x)$ for all $x \in [o, b]$. And since 
        $-\bigints\limits_{b}^{0}=\bigints\limits_{0}^{b}$ we then have,
        \\
        \\
        $
          -\bigints\limits_{+b}^{0} f(-x) dx=\bigints\limits_{0}^{+b} f(x) dx
        $
        \\
        \\
        So, putting this all together we have,
        \\
        \\
        $
          \bigints\limits_{-b}^{+b} f(x) dx
          =\bigints\limits_{0}^{+b} f(x)dx+\bigints\limits_{0}^{+b}
          \\
          \\
          \\
          \therefore ~~~~ \bigints\limits_{-b}^{+b} f(x) dx=2\bigints\limits_{0}^{+b} f(x) dx ~~~~~ \blacksquare 
        $
        \\
        \\
        \\
        \\
        \\
        \\
        Similarly we have,
        \\
        \\
        $
          \bigints\limits_{-b}^{+b} f(x) dx
          =\bigints\limits_{-b}^{0} f(x) dx+\bigints\limits_{0}^{+b} f(x)dx
          =\bigints\limits_{0}^{+b} f(-x) dx+\bigints\limits_{0}^{+b} f(x) dx
        $
        \\
        \\
        Since $f$ is an odd function, we have $f(-x)=-f(x)$ for all $x \in [0, b]$. Thus,
        \\
        \\
        $
          \bigints\limits_{0}^{+b} f(-x) dx
          =-\bigints\limits_{0}^{+b} f(x) dx
          \\
          \\
          \\
          \therefore ~~~~ \bigints\limits_{-b}^{+b} f(x) dx=0 ~~~~~ \blacksquare
        $
        \\
        \\
      }

    % \item (15) Suppose that $f: R \longrightarrow R$ is periodic and integrable on every closed interval. If $p$ is the
    % period of $f$, prove that for any $a \in R$,
    % $$\bigints\limits_{0}^{p} f dx=\bigints\limits_{a}^{a+p} f dx$$

    %   % \textcolor{hwColor}{
        
    %   % }

  \end{enumerate}


  \textbf{5.4 THE FUNDAMENTAL THEOREM OF INTEGRAL CALCULUS}
  \begin{enumerate}
    \item (16) Use the Fundamental Theorem of Integral Calculus to compute the following:
    
    \textcolor{hwColor}{
      \\
      \\
      Let's recall what the Fundamental Theorems of Integral Calculus. The first fundamental theorem
      of calculus states that, if $f$ is continuous on the closed interval 
      $[a,b]$ and $F$ is the indefinite integral of $f$ on $[a,b]$, then
      $$\bigints\limits_{a}^{b} f(x) dx=F(b)-F(a)$$
      This result, while taught early in elementary calculus courses, is actually a very deep result connecting the 
      purely algebraic indefinite integral and the purely analytic (or geometric) definite integral.
      \\
      \\
      The second fundamental theorem of calculus holds for $f$ a continuous function on an open interval $I$ and a any 
      point in $I$, and states that if $F$ is defined by
      $$F(x)=\bigints\limits_{a}^{b} f(x)dx$$
      then $F^'(x)=f(x)$ at each point in $I$.
    }

    \begin{enumerate}
      \item $\bigints\limits_{0}^{3} \left(x^2-x\right) dx$

        \textcolor{hwColor}{
          \\
          $
            \bigints\limits_{0}^{3} \left(x^2-x\right) dx
            =\bigints\limits_{0}^{3} x^2 dx-\bigints\limits_{0}^{3} x dx
            =\left[\dfrac{x^3}{3}-\dfrac{x^2}{2}\right]\Big|_{0}^{3}
            \\
            \\
            \\
            =\left[\dfrac{3^3}{3}-\dfrac{3^2}{2}\right]-\left[\dfrac{0^3}{3}-\dfrac{0^2}{2}\right]
            \\
            \\
            \\
            \therefore ~~~~ \bigints\limits_{0}^{3} \left(x^2-x\right) dx=\dfrac{9}{2} ~~~~ \checkmark
            \\
          $
        }

      \item $\bigints\limits_{-2}^{4} \left(1-x^3-x^2\right) dx$

        \textcolor{hwColor}{
          \\
          $
            \bigints\limits_{-2}^{4} \left(1-x^3-x^2\right) dx
            =\bigints\limits_{-2}^{4} dx- \bigints\limits_{-2}^{4} x^3 dx-\bigints\limits_{-2}^{4} x^2 dx
            =\left[x-\dfrac{x^4}{4}-\dfrac{x^3}{3}\right]\Big|_{-2}^{4}
            \\
            \\
            \\
            =\left[4-\dfrac{4^4}{4}-\dfrac{4^3}{3}\right]-\left[-2-\dfrac{(-2)^4}{4}-\dfrac{(-2)^3}{3}\right]
            \\
            \\
            \\
            \therefore ~~~~ \bigints\limits_{-2}^{4} \left(1-x^3-x^2\right) dx=-78 ~~~~ \checkmark
          $
          \\
        }

      \item $\bigints\limits_{0}^{\dfrac{\pi}{2}} x sin(x^2) dx$

        \textcolor{hwColor}{
          \\
          \\
          For this integral we can use the U-Substitution method. $u=x^2 \Longrightarrow du=2x dx$
          \\
          \\
          Since $u=x^2$ then we have: $\begin{cases}
            x=0 \longrightarrow u=0
            \\
            \\
            x=\dfrac{\pi}{2} \Longrightarrow u=\dfrac{\pi^2}{4}
          \end{cases}$
          \\
          \\
          $
            \bigints\limits_{0}^{\dfrac{\pi}{2}} x sin(x^2) dx
            =\bigints\limits_{0}^{\dfrac{\pi^2}{4}} x sin(u) \dfrac{1}{2x} du
            =\dfrac{1}{2} \bigints\limits_{0}^{\dfrac{\pi^2}{4}} sin(u)du
            \\
            \\
            \\
            =\dfrac{1}{2} \left[-cos(u)\right]\Big|_{0}^{\dfrac{\pi^2}{4}}
            =\dfrac{1}{2} \left[\left(-cos(\dfrac{\pi^2}{4})\right)-\left(-cos(0)\right)\right]
            \\
            \\
            \\
            =\dfrac{1}{2} \left[-cos(\dfrac{\pi^2}{4})+1\right]
            \\
            \\
            \\
            \therefore ~~~~ \bigints\limits_{0}^{\dfrac{\pi}{2}} x sin(x^2) dx \approx 0.8906 ~~~~\checkmark
          $
        }

    \end{enumerate}

    \item (17) Define $f: [0, 2] \longrightarrow R$ by $f(x)=2x-x^2$ for $0 \leq x \leq 1$ and $f(x)=(x-2)^2$
    for $1 \leq x \leq 2$. Prove that $f$ is integrable on $[0, 2]$ and find the integral of $f$ over $[0, 2]$.
    Do not use Theorem $5.10$, but rather find the integral by methods similar to those used in the proof
    of Theorem $5.8$.

      \textcolor{hwColor}{
        \\
        In fact, when mathematicians say that a function is integrable, they mean only that the integral is well 
        defined — that is, that the integral makes mathematical sense. In practical terms, integrability hinges on 
        continuity: If a function is continuous on a given interval, it’s integrable on that interval. 
        \\
        \\
        The given functions are continuous since they are also defined at 1. $f(1)=1$ and $g(1)=1$
        \\
        \\
        We know that 
        $
          \begin{cases}
            \bigints f(x) dx=\bigints \left(2x-x^2\right) dx=x^2-\dfrac{x^3}{3}+C
            \\
            \\
            \bigints g(x) dx=\bigints  (x-2)^2 dx=\dfrac{x^3}{3}-2x^2+4x=\dfrac{\left(x-2\right)^3}{3}+C
          \end{cases}
          \\
          \\
          \\
          \bigints\limits_{0}^{1} f(x) dx+\bigints\limits_{1}^{2} g(x) dx
          =\left[F(1)-F(0)\right]+\left[G(2)-G(1)\right]
          =\left[\dfrac{2}{3}-0\right]+\left[\dfrac{8}{3}-\dfrac{7}{3}\right]
          =1 ~~~~ \checkmark
        $
        \\
      }

  \end{enumerate}

  \textbf{5.5 ALGEBRA OF INTEGRABLE FUNCTIONS}
  \begin{enumerate}
    \item (18) Suppose $f$ and $g$ are differentiable on $[a, b]$ and $f^'$ and $g^'$ are integrable on $[a, b]$.
    Prove that $f^' g$ and $g^' f$ are integrable on $[a, b]$ and that
    $$\bigints\limits_{a}^{b} f^' g dx=f(b) g(b)-f(a) g(a)-\bigints\limits_{a}^{b} g^' f dx$$
    Of course, this is the \emph{integration-by-parts} formula.

      \textcolor{hwColor}{
        \\
        Let's define a new function $w(x)=f(x) g(x)$.
        \\
        \\
        $
          w(x)=f(x) g(x)
          \\
          \\
          w^'(x)=f^'(x) g(x)+f(x) g^'(x)
          \\
          \\
          \\
          \bigints\limits_{a}^{b} w^'(x) dx=\bigints\limits_{a}^{b} \left[f^'(x) g(x)+f(x) g^'(x)\right] dx
          \\
          \\
          \\
          \bigints\limits_{a}^{b} w^'(x) dx=\bigints\limits_{a}^{b} f^'(x) g(x) dx+\bigints\limits_{a}^{b} f(x) g^'(x) dx
        $
        \\
        \\
        With the help of the Fundamental Theorem of Integral Calculus we can conclude,
        \\
        \\
        $
          \bigints\limits_{a}^{b} w^'(x) dx=f(b)g(b)-f(a)g(a)
          \\
          \\
          \\
          \therefore ~~~~ f(b)g(b)-f(a)g(a)=\bigints\limits_{a}^{b} f^'(x) g(x) dx+\bigints\limits_{a}^{b} f(x) g^'(x) dx
          \\
          \\
          \\
          \therefore ~~~~ \bigints\limits_{a}^{b} f^'(x) g(x) dx=f(b)g(b)-f(a)g(a)-\bigints\limits_{a}^{b} f(x) g^'(x) dx ~~~~ \checkmark
        $
      }

  \end{enumerate}


  \textbf{5.6 DERIVATIVES OF INTEGRALS}
  \begin{enumerate}
    \item (30) Find $f^'$ where $f$ is defined on $[0, 1]$ as indicated:
    
      \textcolor{hwColor}{
        \\
        $
          I(x,t)=\bigints\limits_{a(t)}^{b(t)} f(x,t) dx
          \\
          \\
          \\
          \dfrac{d}{dt} I(x,t)=\lim\limits_{\Delta t \to 0} \dfrac{I(x, t+t\Delta)-I(x, t)}{\Delta t}
          =\lim\limits_{\Delta t \to 0} \dfrac{1}{\Delta t} \left[\bigints\limits_{a+\Delta a}^{b+\Delta b} f(x, t+\Delta t) dx-\bigints\limits_{a}^{b} f(x, t)dx\right]
          \\
          \\
          \\
          =\lim\limits_{\Delta t \to 0} \dfrac{1}{\Delta t} \left[
            \bigints\limits_{a+\Delta a}^{a} f(x, t+\Delta t) dx
            +\bigints\limits_{a}^{b} f(x, t+\Delta t) dx
            +\bigints\limits_{a}^{b+\Delta b} f(x, t+\Delta t) dx
            -\bigints\limits_{a}^{b} f(x, t) dx
          \right]
          \\
          \\
          \\
          =\lim\limits_{\Delta t \to 0} \left[
            \bigints\limits_{a}^{b} \left(f(x, t+\Delta t)-f(x,t)\right) dx
            +\bigints\limits_{b}^{b+\Delta b} f(x, t+\Delta t) dx
            -\bigints\limits_{a}^{a+\Delta a} f(x, t+\Delta t) dx
          \right]
        $
        \\
        \\
        By using the Mean-Value theorem we have,
        \\
        \\
        $
          \bigints\limits_{b}^{b+\Delta b} f(x, t+\Delta t) dx=F(x, b+\Delta b)-F(x,b)
          \\
          \\
          \lim\limits_{\Delta t \to 0} f(c, t) \Delta b=f(b, t) \Delta b
        $
        \\
        \\
        Getting back to our original limit we have,
        \\
        \\
        $
          \lim\limits_{\Delta t \to 0} \left[
            \dfrac{\bigints\limits_{a}^{b} f(x, t+\Delta t)-f(x,t) dx}{\Delta t}
            +\dfrac{f(b, t) \Delta b}{\Delta t}
            -\dfrac{f(a, t) \Delta a}{\Delta t}
          \right]
          \\
          \\
          \\
          \\
          \Longrightarrow \dfrac{d}{dt} I(x, t)=\bigints\limits_{a}^{b} \partial_t f(x, t)dx+f(b, t)\dfrac{db}{dt}-f(a, t)\dfrac{da}{dt}
          \\
          \\
          \\
          \\
          \therefore ~~~~ \dfrac{d}{dt} \bigints\limits_{a}^{b} f(x, t) dx=\bigints\limits_{a}^{b} \partial_t f(x, t) dx ~~~~ \checkmark
          \\
          \\
        $
        The above equation is known as the \textbf{\emph{Leibniz rule for integrals}}. We can use it for the following integrals.
        \\
        \\
      }

    \begin{enumerate}
      \item $f(x)=\bigints\limits_{0}^{x} \sqrt{t^2+1} dt$

        \textcolor{hwColor}{
          $
            \dfrac{d}{dx} f(x, t)=\bigints\limits_{a}^{b} \partial_x f(x, t) dt+f(b, t) \dfrac{db}{dx}-f(a, t) \dfrac{da}{dx}
            \\
            \\
            \\
            f(x, t)=\sqrt{t^2+1}
            \\
            \\
            \therefore ~~~~ \dfrac{d}{dx} f(x, t)=\bigints\limits_{0}^{x} 0 dt+f(x, t) \dfrac{dx}{dx}-f(0, t) \dfrac{d 0}{dx}
            \\
            \\
            \\
            \therefore ~~~~ \dfrac{d}{dx} f(x, t)=0+\sqrt{x^2+1}-0
            \\
            \\
            \\
            \therefore ~~~~ \dfrac{d}{dx} f(x, t)=\sqrt{x^2+1} ~~~~ \checkmark
            \\
            \\
            \rule{15cm}{2pt}
            \\
            \\
          $
        }

      \item $f(x)=\bigints\limits_{x}^{1} cos\left(\dfrac{1}{t+1}\right) dt$

        \textcolor{hwColor}{
          $
            \dfrac{d}{dx} f(x, t)=\bigints\limits_{a}^{b} \partial_x f(x, t) dt+f(b, t) \dfrac{db}{dx}-f(a, t) \dfrac{da}{dx}
            \\
            \\
            \\
            f(x, t)=cos\left(\dfrac{1}{t+1}\right)
            \\
            \\
            \\
            \therefore ~~~~ \dfrac{d}{dx} f(x, t)=\bigints\limits_{x}^{1} 0 dt+f(1, t)\dfrac{d}{dx} 1-f(x, t) \dfrac{dx}{dx}
            \\
            \\
            \\
            \therefore ~~~~ \dfrac{d}{dx} f(x, t)=0+0-cos\left(\dfrac{1}{x+1}\right)
            \\
            \\
            \\
            \therefore ~~~~ \dfrac{d}{dx} f(x, t)=-cos\left(\dfrac{1}{x+1}\right) ~~~~ \checkmark
            \\
            \\
          $
        }

      \item $f(x)=\bigints\limits_{x^2}^{2x} sin(t^2) dt$

        \textcolor{hwColor}{
          $
            \dfrac{d}{dx} f(x, t)=\bigints\limits_{a}^{b} \partial_x f(x, t) dt+f(b, t) \dfrac{db}{dx}-f(a, t) \dfrac{da}{dx}
            \\
            \\
            \\
            f(x, t)=sin(t^2)
            \\
            \\
            \\
            \therefore ~~~~ \dfrac{d}{dx} f(x, t)=\bigints\limits_{x^2}^{2x} 0 dt+f(2x, t) \dfrac{d}{dx}(2x)-f(x^2, t) \dfrac{d}{dx} (x^2)
            \\
            \\
            \\
            \therefore ~~~~ \dfrac{d}{dx} f(x, t)=0+sin\left[\left(2x\right)^2\right] \times 2-sin\left[\left(x^2\right)^2\right] \times 2x
            \\
            \\
            \\
            \therefore ~~~~ \dfrac{d}{dx} f(x, t)=2 sin(4x^2)-2x sin(x^4) ~~~~ \checkmark
            \\
            \\
            \rule{15cm}{2pt}
            \\
            \\
          $
        }

      \item $f(x)=\bigints\limits_{x}^{\sqrt{x}} \dfrac{1}{1+t^3} dt$

        \textcolor{hwColor}{
          $
            \dfrac{d}{dx} f(x, t)=\bigints\limits_{a}^{b} \partial_x f(x, t) dt+f(b, t) \dfrac{db}{dx}-f(a, t) \dfrac{da}{dx}
            \\
            \\
            \\
            f(x,t)=\bigints\limits_{x}^{\sqrt{x}} \dfrac{1}{1+t^3} dt
            \\
            \\
            \\
            \therefore ~~~~ \dfrac{d}{dx} f(x, t)=\bigints\limits_{x}^{\sqrt{x}} 0 dt+f(\sqrt{x}, t) \dfrac{d}{dx} \sqrt{x}-f(x, t) \dfrac{dx}{dx}
            \\
            \\
            \\
            \therefore ~~~~ \dfrac{d}{dx} f(x, t)=0+\dfrac{1}{1+\sqrt{3}^3}\dfrac{1}{2\sqrt{x}}-\dfrac{1}{1+x^3}
            \\
            \\
            \\
            \therefore ~~~~ \dfrac{d}{dx} f(x, t)=\dfrac{1}{2\sqrt{x} \left(1+\sqrt{3}^3\right)}-\dfrac{1}{1+x^3} ~~~~ \checkmark
            \\
            \\
          $
        }

    \end{enumerate} 

  \end{enumerate}


  \textbf{5.7 MEAN-VALUE AND CHANGE-OF-VARIABLE THEOREMS}
  \begin{enumerate}
    \item (33) Suppose $f: R \longrightarrow R$ is continuous and has period $p$, so that $f(x+p)=f(x)$ for all $x \in R$.
    Show that $\bigints\limits_{x}^{x+p} f(t) dt$ is independent of $x$ in that, for all $x, y,$
    $$\bigints\limits_{x}^{x+p} f(t) dt=\bigints\limits_{y}^{y+p} f(t) dt$$
    Show, then, that $\bigints\limits_{0}^{p}  \left[f(x+a)-f(x)\right]dx=0$ for any real number $a$. Conclude that for any
    real number $a$, there is $x$ such that $f(x+a)=f(x)$.

      \textcolor{hwColor}{
          Behnam was here
      }

    \item (35) Use Theorem $5.18$ to evaluate the integrals:
    
      \textcolor{hwColor}{
        Theorem $5.18$ states that $\phi: [a, b] \longrightarrow R$ is differentiable and $\phi^'$ is continuous.
        Further assume that $\phi \left([a, b]\right)=[c, d]$ with $\phi(a)=c$ and $\phi(b)=d$. If 
        $f: [c, d] \longrightarrow R$ is continuous, then
        $$
          \bigints\limits_{a}^{b} f\left( \phi(t)\right) \phi^'(t) dt=\bigints\limits_{c}^{d} f(x) dx
        $$ 
      }
    
    \begin{enumerate}
      \item $\bigints\limits_{0}^{3} \sqrt[3]{1+x^2} xdx$

        \textcolor{hwColor}{
          $
            I=\bigints\limits_{0}^{3} \sqrt[3]{1+x^2} xdx
            \\
            \\
            u=1+u^2 \Longrightarrow \dfrac{du}{dx}=2x, ~~~
            \begin{cases}
              x=0 \Longrightarrow u=1
              \\
              \\
              x=3 \Longrightarrow u=10
            \end{cases}
            \\
            \\
            \\
            \therefore ~~~~ I=\bigints\limits_{1}^{10} \sqrt[3]{u} x \dfrac{du}{2x}=\dfrac{1}{2} \dfrac{3}{4} u^{\dfrac{3}{4}}\Big|_{1}^{10}
            \\
            \\
            \\
            \therefore ~~~~ I=\dfrac{3}{8} \left(10^{\dfrac{3}{4}}-1\right) ~~~~ \checkmark
            \\
          $
        }

      \item $\bigints\limits_{1}^{4} \dfrac{\left(\sqrt{x}+2\right)^3}{\sqrt{x}} dx$

        \textcolor{hwColor}{
          $
            I=\bigints\limits_{1}^{4} \dfrac{\left(\sqrt{x}+2\right)^3}{\sqrt{x}} dx
            \\
            \\
            u=\sqrt{x}+2 \Longrightarrow \dfrac{du}{dx}=\dfrac{1}{2 \sqrt{x}}, ~~~
            \begin{cases}
              x=1 \Longrightarrow u=3
              \\
              \\
              x=4 \Longrightarrow u=4
            \end{cases}
            \\
            \\
            \\
            \therefore ~~~~ I=\bigints\limits_{3}^{4} \dfrac{u^3}{\sqrt{x}} 2\sqrt{x} du=\dfrac{u^4}{2}\Big|_{3}^{4}
            \\
            \\
            \\
            \therefore ~~~~ I=\dfrac{175}{2} ~~~~ \checkmark
            \\
          $
        }

      \item $\bigints\limits_{1}^{\sqrt{3}} \dfrac{\sqrt{x^2-9}}{x} dx$

        \textcolor{hwColor}{
          $
            I=\bigints\limits_{1}^{\sqrt{3}} \dfrac{\sqrt{x^2-9}}{x} dx
            \\
            \\
            u=\sqrt{x^2-9} \Longrightarrow \dfrac{du}{dx}=\dfrac{x}{\sqrt{x^2-9}}
            \\
            \\
            k=\bigints \dfrac{u^2}{u^2+9} du
            =\bigints \left(\dfrac{u^2+9}{u^2+9}-\dfrac{9}{u^2+9}\right) du
            =\bigints du-\bigints \dfrac{9}{u^2+9} du
            =u-w
            \\
            \\
            \\
            w=\bigints \dfrac{9}{u^2+9} du
            \\
            \\
            g=\dfrac{u}{3} \Longrightarrow \dfrac{dg}{du}=\dfrac{1}{3}
            \\
            \\
            \\
            w=\bigints \dfrac{3}{9(g^2+1)} dg=\dfrac{1}{3} arctan(g)=\dfrac{1}{3} arctan(\dfrac{u}{3})
            \\
            \\
            \\
            \therefore ~~~~ k=u-w=u-\dfrac{1}{3} arctan(\dfrac{u}{3})
            \\
            \\
            \\
            \therefore ~~~~ k=\sqrt{x^2-9}-\dfrac{1}{3} arctan(\dfrac{\sqrt{x^2-9}}{3})+C
            \\
            \\
          $
          There is one big issue right here, $\dfrac{\sqrt{x^2-9}}{x}$ is not defined at $1$ and $\sqrt{3}$,
          therefore we can not find a numerical value for the given integral at the wanted domain.
          \\
        }

      \item $\bigints\limits_{0}^{1} \dfrac{x^2}{\sqrt{1-x^2}} dx$

        \textcolor{hwColor}{
          $
            I=\bigints\limits_{0}^{1} \dfrac{x^2}{\sqrt{1-x^2}} dx
            \\
            \\
            x=sin(u) \Longrightarrow dx=cos(u) du
            \\
            \\
            k=\bigints \dfrac{sin^2(u)}{\sqrt{1-sin^2(u)}} cos(u) du
            =\bigints sin^2(u) du
            \\
            \\
            \\
            k=\dfrac{arcsin(x)-x \sqrt{1-x^2}}{2}+C
            \\
            \\
            \\
            \therefore ~~~~ I=\dfrac{arcsin(x)-x \sqrt{1-x^2}}{2} \Big|_{0}^{1}
            \\
            \\
            \\
            \therefore ~~~~ I=\dfrac{\pi}{4} ~~~~ \checkmark
            \\
          $
        }

    \end{enumerate}
    [You may want to consult your calculus book before trying (c) and (d).] 


  \end{enumerate}

  \pagebreak

  \textbf{6.1 CONVERGENCE OF INFINITE SERIES}
  \begin{enumerate}
    \item (1) Let $\{ a_n \}_{n=1}^{\infty}$ be a sequence of real numbers. Prove that
    $$\sum\limits_{n=1}^{\infty} (a_n-a_{n+1})$$
    converges iff $\{ a_n \}_{n=1}^{\infty}$ converges. If $\sum\limits_{n=1}^{\infty} (a_n-a_{n+1})$ converges, 
    what is the sum?

      % \textcolor{hwColor}{
          
      % }

    \item (3) Prove that $\sum\limits_{n=1}^{\infty} 2^n r^n$ converges if $|r| < \dfrac{1}{2}$ and find the sum. 

      \textcolor{hwColor}{
        $
          S_n=\sum\limits_{n=1}^{\infty} 2^n r^n
          =\sum\limits_{n=1}^{\infty}=2r \times \dfrac{1-(2r)^k}{1-2r}
          \\
          \\
          \\
          \lim\limits_{n \to \infty} 2r \times \dfrac{1-(2r)^n}{1-2r}=\dfrac{2r}{1-2r}
          \\
          \\
        $
        $|r|>\dfrac{1}{2}$, therefore the series converges. $~~~~ \checkmark$
        \\
      }

    \item (4) Prove that the series $\sum\limits_{n=0}^{\infty} 3^{-n}$ converges and find the limit. 

      \textcolor{hwColor}{
        $
          S_n=\sum\limits_{n=0}^{\infty} 3^{-n}=\dfrac{1-\left(\dfrac{1}{3}\right)^{k+1]}}{1-\dfrac{1}{3}}
          \\
          \\
          \\
          \therefore ~~~~ S_n=\dfrac{3-3^{-k}}{2}
          \\
          \\
          \\
          \lim\limits_{n \to \infty} \dfrac{3-3^{-n}}{2}=\dfrac{3}{2}
        $
        \\
        \\
        Again like the previous qeustion, since $|r|<1$, the series converges. Note that the limit of the sequence is zero though.
      }

    % \item (6) Use induction to show that $1+\dfrac{1}{\sqrt{2}}+...+\dfrac{1}{\sqrt{n}} \geq \sqrt{n}$ for $n \geq 1$.
    % Can you use this fact to determine whether the series $\sum\limits_{n=1}^{\infty} \dfrac{1}{\sqrt{n}}$ converges or diverges? 
 
    %   % \textcolor{hwColor}{
          
    %   % }

  \end{enumerate}

  \pagebreak

  \textbf{6.2 ABSOLUTE CONVERGENCE AND THE COMPARISON TEST}
  \begin{enumerate}
    \item (13) Suppose $\sum\limits_{n=1}^{\infty} a_n$ converges absolutely and $\{ b_n \}_{n=1}^{\infty}$ is bounded.
    Prove that $\sum\limits_{n=1}^{\infty} a_n b_n$ converges absolutely.

      \textcolor{hwColor}{
        Since $\sum\limits_{n=1}^{\infty} a_n$ converges and $\{ b_n \}_{n=1}^{\infty}$ is bounded, 
        there is $M \in R$ with $M>0$ such that $|b_n|\leq M$ for any $n$. 
        \\
        \\
        Let $\epsilon >0$, then we have $\dfrac{\epsilon}{M}>0$. 
        \\
        \\
        $
          m \geq n > N
          \\
          \\
          | \sum\limits_{k=n}^{m} a_k | < \dfrac{\epsilon}{M}
        $
        By using the triangle inequality we have,
        \\
        \\
        $
          | \sum\limits_{k=n}^{m} a_k b_k| \leq   \sum\limits_{k=n}^{m} a_k b_k
          =\sum\limits_{k=n}^{m} |a_k| |b_k| \leq M \sum\limits_{k=n}^{m} |a_k| < M \times \dfrac{\epsilon}{M}
          \\
          \\
          \\
        $
        Therefore, $\sum\limits_{n=1}^{\infty} a_n b_n$ converges absolutely. 
        \\
      }

    \item (15) Determine which of the following infinite series converge.
    \begin{enumerate}
      \item $\sum\limits_{n=1}^{\infty} \dfrac{1}{(n+1)(2n-1)}$

        \textcolor{hwColor}{
          \\
          \\
          $
            \sum\limits_{n=1}^{m} \dfrac{1}{(n+1)(2n-1)}
            =\sum\limits_{n=1}^{m} \left(\dfrac{1}{n+1}-\dfrac{1}{n+2}\right)
            =\sum\limits_{n=1}^{m} \dfrac{1}{n+1}-\sum\limits_{n=1}^{m}\dfrac{1}{n+2}
            \\
            \\
            \\
            \therefore ~~~~  \sum\limits_{n=1}^{m} \dfrac{1}{(n+1)(2n-1)}=1-\dfrac{1}{n+2}
            \\
            \\
            \\
            \therefore ~~~~ \lim\limits_{m \to \infty} \sum\limits_{n=0}^{m} \dfrac{1}{(n+1)(2n-1)}
            =1-\lim\limits_{m \to \infty}
            \\
            \\
            \\
            \therefore ~~~~ \sum\limits_{n=1}^{\infty} \dfrac{1}{(n+1)(2n-1)}=1
          $
          \\
          \\
          Hence, the series converges. $~~~~ \checkmark$
          \\
          \\
        }

      \item $\sum\limits_{n=1}^{\infty} \dfrac{n}{n+1}$

        \textcolor{hwColor}{
          $
            \\
            \dfrac{n}{n+1}=\dfrac{1}{1+\dfrac{1}{n}}
            \\
            \\
            \\
            \lim\limits_{n \to \infty} \dfrac{1}{1+\dfrac{1}{n}}=1
            \\
            \\
          $
          Since the result of the above limit is $\neq 0$ then the series diverges.
          \\
        }

      \item $\sum\limits_{k=1}^{\infty} \dfrac{2}{3k}$
      
        \textcolor{hwColor}{
          $
            \sum\limits_{k=1}^{\infty} \dfrac{2}{3k}
            =\dfrac{2}{3} \sum\limits_{k=1}^{\infty} \dfrac{1}{k}
            \\
            \\
            \\
          $
          For this case if we allow the use of Cauchy's condition, then the given 
          series diverges.
          \\
          \\
        }

      \item $\sum\limits_{m=1}^{\infty} \dfrac{\sqrt{m+1}-\sqrt{m}}{m}$

        \textcolor{hwColor}{
          \\
          $
            \sum\limits_{m=1}^{\infty} \dfrac{\sqrt{m+1}-\sqrt{m}}{m}
            \\
            \\
            =\sum\limits_{m=1}^{\infty} \dfrac{\sqrt{m+1}-\sqrt{m}}{m} \times \dfrac{\sqrt{m+1}+\sqrt{m}}{\sqrt{m+1}+\sqrt{m}}
            \\
            \\
            =\sum\limits_{m=1}^{\infty} \dfrac{1}{m\left[\sqrt{m+1}+\sqrt{m}\right]}
          $
          \\
          \\
          \textbf{Comparsion:}
          \\
          \\
          By comparing the given series with $\sum\limits_{m=1}^{\infty} n^{-\dfrac{3}{2}}$ which we know is a convergent series we 
          can conclude that $\sum\limits_{m=1}^{\infty} \dfrac{\sqrt{m+1}-\sqrt{m}}{m}$ is convergent.
          \\
        }

      \item $\sum\limits_{m=1}^{\infty} \dfrac{3^m}{5^{m+1}}$

        \textcolor{hwColor}{
          \\
          \\
          $
            \sum\limits_{m=1}^{\infty} \dfrac{3^m}{5^{m+1}}
            =\sum\limits_{m=1}^{\infty} \dfrac{3^m}{5^m \times 5}
            \\
            \\
            \\
            =\dfrac{1}{5} \sum\limits_{m=1}^{\infty} \dfrac{3^m}{5^m}
            =\dfrac{1}{5} \sum\limits_{m=1}^{\infty} \left(\dfrac{3}{5}\right)^m
          $
          \\
          \\
          \\
          The given series is convergent because $|r| <1$.
          \\
          \\
        }

      % \item $\sum\limits_{m=1}^{\infty} \dfrac{1}{m^2} \left(\dfrac{100m+1}{m}\right)$

      %   % \textcolor{hwColor}{
          
      %   % }

    \end{enumerate}

    \item (16) (Limit-comparison test.) Prove the following generalization of Theorem $6.5$. Suppose
    $\sum\limits_{n=0}^{\infty} a_n$ and $\sum\limits_{n=0}^{\infty} b_n$ are series of positive terms such that 
    $$\{ \dfrac{a_n}{b_n} \}_{n=1}^{\infty}$$
    converges to $L\neq 0$. Then $\sum\limits_{n=0}^{\infty} a_n$ and $\sum\limits_{n=0}^{\infty} b_n$  either both diverge or both converge.
    What can be concluded if $L=0$

      \textcolor{hwColor}{
        \\
        \\
        \\
      }
 
  \end{enumerate}


  \textbf{6.3 RATIO AND ROOT TESTS}

    \textcolor{hwColor}{
      \\
      \\
      Let's just have a quick review. When we have a $\sum a_n$. Define,
      $$
        L=\lim\limits_{n \to \infty} |\dfrac{a_{n+1}}{a_n}|
      $$
      Then,
      \\
      \begin{itemize}
        \item If $L<1$ the series is absolutely convergent (and hence convergent).
        \item If $L>1$ the series is divergent.
        \item If $L=1$ the series may be divergent, conditionally convergent, or absolutely convergent.
      \end{itemize}
    }

  \begin{enumerate}
    \item (19) Use Theorem 6.9 to determine the values of $r$ for which $\sum\limits_{n=0}^{\infty} n r^n$ converges.

      \textcolor{hwColor}{
        $ 
          \\
          L=\lim\limits_{n \to \infty} |\dfrac{(n+1) r^{n+1}}{n r^n}|
          =\lim\limits_{n \to \infty} |r \left(1+n^{-1}\right)|
        $
        \\
        \\
        By taking the limit we get that $L>|r|$
        \\
      }

    \item (20) Prove that $\{ n x^n\}_{n=1}^{\infty}$ converges to zero if $|x| < 1$.

      \textcolor{hwColor}{
        \\
        $
          L=\lim\limits_{n \to \infty} |\dfrac{a_{n+1}}{a_n}|
          =\lim\limits_{n \to \infty} |\dfrac{(n+1) x^{n+1}}{n x^n}|
          \\
          \\
          \\
          =\lim\limits_{n \to \infty} |x\left(1+n^{-1}\right)|=x
        $
        \\
        \\
        This series must converge because $|x|<1$.
      }

  \end{enumerate}


  \textbf{6.5 POWER SERIES}
  \begin{enumerate}
    \item (34) Show that $\sum\limits_{n=1}^{\infty} n! x^n$ converges only for $x=0$.

      % \textcolor{hwColor}{
          
      % }

    \item (35) Show that $\sum\limits_{n=1}^{\infty} \dfrac{x^n}{n}$ converges iff $-1 \leq x <1$.

      % \textcolor{hwColor}{
          
      % }

    \item (38) Show that the power series $\sum\limits_{n=0}^{\infty} a_n x^n$ and $\sum\limits_{n=1}^{\infty} n a_n x^{n-1}$ either both converge for all 
    $x$, both converge only for $x=0$, or both have the same finite nonzero radius of convergence.

      % \textcolor{hwColor}{
          
      % }

  \end{enumerate}


  \textbf{6.6 TAYLOR SERIES}
  \begin{enumerate}
    \item (42) Write the Taylor series for $log ~ x$ using powers $x-1$. See project $5.1$ in Chapter 5 if
    you have forgotten the derivative of $log ~ x$. Prove that this series converges to $log ~ x$ for $1 \leq x <2$.

      % \textcolor{hwColor}{
          
      % }

    \item (43) Write the Taylor series for $\sqrt{1-x}$ in power of $x$. Prove that the series converges to $\sqrt{1-x}$ for
    $-1 < x \leq 0$.

      % \textcolor{hwColor}{
          
      % }

    \item (44) If $x > 0$, show that $|\left(1+x\right)^{\dfrac{1}{3}}-\left(1+\dfrac{x}{3}-\dfrac{x^2}{9}\right)| \leq \dfrac{5x^3}{81}$.

      % \textcolor{hwColor}{
          
      % }

  \end{enumerate}

\end{document}
